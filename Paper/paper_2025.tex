\documentclass[12pt]{article}

\usepackage{tgtermes}
\usepackage{epsf}
\usepackage{epstopdf}
\usepackage{amsmath}
\usepackage{graphicx}
\usepackage{booktabs}
\usepackage[colorlinks=true,linkcolor=blue,citecolor=blue]{hyperref}
\usepackage{dcolumn}
\usepackage{amsmath, amsthm, amssymb}
\usepackage{mwe}
\usepackage{url}
%\usepackage{harvard}
\usepackage{fancyheadings}
\usepackage{longtable}
\usepackage{authblk}
\usepackage{setspace}
%\usepackage[nomarkers]{endfloat}
\usepackage{float}
\usepackage{bbm}
%\usepackage{titling}
\usepackage{subcaption}
\usepackage{algorithm}
\usepackage{algorithmic}
\usepackage{import}
\usepackage[backend=biber,style=authoryear,
sorting=nyt,citestyle=authoryear]{biblatex}
\addbibresource{papercitations.bib}
%\usepackage[nomarkers,nofiglist,notablist]{endfloat}
\usepackage{subcaption}
\usepackage{caption}

\onehalfspacing
\textwidth 6.5in \oddsidemargin 0in \evensidemargin -0.6in
\textheight 8.5in \topmargin -0.2in

\newcolumntype{L}[1]{>{\raggedright\let\newline\\
		\arraybackslash\hspace{0pt}}m{#1}}
\newcolumntype{C}[1]{>{\centering\let\newline\\
		\arraybackslash\hspace{0pt}}m{#1}}
\newcolumntype{R}[1]{>{\raggedleft\let\newline\\
		\arraybackslash\hspace{0pt}}m{#1}}
\newcolumntype{P}[1]{>{\raggedright\tabularxbackslash}p{#1}}

\newtheorem{theorem}{Theorem}[section]
\newtheorem{corollary}[theorem]{Corollary}
\newtheorem{proposition}[theorem]{Proposition}
\newtheorem{lemma}[theorem]{Lemma}

\captionsetup{justification=centering,singlelinecheck=false}


\newcommand{\xsub}[1]{%
	\mbox{\scriptsize\begin{tabular}{@{}c@{}}#1\end{tabular}}%
}

%\renewcommand{\thetable}{\Roman{table}}

\begin{document}
	
	
	
	
	\linespread{1.2}\title{\vspace{-0.5in} Non-Merger Affiliations in Already Consolidated Health Care Markets} 
	
	\date{\today}
	
	\author{\vspace{10mm}Hanna Glenn\footnote{Department of Economics, Emory University, 1602 Fishburne Drive, Atlanta, GA 30322, hanna.glenn@emory.edu.} }
	
	\maketitle
	%\setlength{\droptitle}{-10pt}
	
	\vspace{-0.2in}
	
	\singlespacing\maketitle


 \vspace{3mm}
	
    \begin{abstract}
		{\small

		} 
	\end{abstract}
	
	
	
	
	

	
	\onehalfspacing
	
	\newpage

    \section{Introduction}

    Market concentration has increased across industries in the last two decades (\cite{grullon2019us}), sparking renewed debate and, more recently, a shift toward stricter antitrust enforcement. Modern antitrust policy primarily focuses on mergers and acquisitions among large firms. Since the Hart–Scott–Rodino Act of 1976, the U.S. government has reviewed certain transactions that may threaten competition, defined by regulators as “a process of rivalry that incentivizes businesses to offer lower prices, improve wages and working conditions, enhance quality and resiliency, innovate, and expand choice” (\cite{DOJFTC2023MergerGuidelines}). Formal challenges remain rare: since 2015, roughly 3\% of proposed mergers have faced extended review, and 32 proposals were dissolved or restructured (\cite{ConversableEconomist2025Antitrust}).

    Careful scrutiny of large-scale mergers is important. However, growing empirical and policy concern centers on unreported, small-scale mergers and non-merger affiliations. Research shows that small-scale mergers, often called “stealth mergers”, can reduce competition (\cite{majerovitz2021consolidation}; \cite{aggarwal2025stealth}; \cite{wollmann2019stealth}; \cite{wollmann2020get}; \cite{kepler2023stealth}). Non-merger affiliations, by contrast, encompass structural ties such as joint ventures, management consolidation, and overlapping boards of directors that do not trigger merger review. Among these, board interlocks are particularly noteworthy because they create direct governance links between firms, are explicitly addressed under the Clayton Act, and have drawn increasing policy attention. While board interlocks are documented (\cite{LemleyVanLoo2025OverlappingDirectors}; \cite{Poberejsky2025InterlocksInnovation}), their competitive effects remain largely hypothesized.

    This issue is especially relevant in health care, where consolidation is a persistent concern and several states have begun regulating small-scale mergers and non-merger affiliations (\cite{ncsl2024healthcare}). Additionally, in September 2025, the FTC announced the resignation of three directors from Sevita Health after uncovering board overlap with Beacon Specialized Living Services, both providers of residential care for individuals with intellectual and developmental disabilities, citing concerns under the Clayton Act and urging firms to proactively avoid such conflicts (\cite{ftc2025sevita}). Board interlocks are no longer abstract governance issues but active antitrust targets. Thus, in this paper, I construct a measure of non-merger affiliation among U.S. nonprofit hospitals that captures both interlocking boards and executive cross-board service, and I examine how these affiliations influence hospital behavior.

    I collect data on nonprofit hospital executives and board members from publicly available IRS Form 990 filings and merge these records with the American Hospital Association (AHA) Annual Survey. Using this combined dataset, I classify hospital affiliations based on system membership and geographic market. I define a non-merger affiliation as the presence of a common board member or an executive serving on another hospital’s board, where the two hospitals are otherwise unaffiliated. That is, they do not belong to the same system or network but share leadership ties. Because the hypothesized effects of non-merger affiliations are anti-competitive, whether the affiliated hospitals compete for the same patients is a critical distinction. To capture this, I separate affiliations into those occurring within the same Hospital Referral Region (HRR) and those across different HRRs, where HRRs represent geographic hospital markets (\cite{Wennberg1973SmallAV}). I show that [X]\% of nonprofit hospitals in my sample share a leader with an otherwise unaffiliated hospital in the same market, and [Y]\% of HRRs contain at least one such affiliation. I consider several hospital outcomes as behaviors that are informed by the corporate governance literature and hospital consolidation literature. I group these outcomes into categories of investment, patient composition, service offerings, and operating costs.

    Similarly to the merger literature, selection into the decision to become affiliated complicates causal inference. I begin with a staggered difference-in-differences design comparing hospitals that form an affiliation to those that do not, and I show that [INSERT FINDINGS]. While event-study estimates reveal no evidence of differential pre-trends, unobserved factors may still influence affiliation decisions. Therefore, my preferred specification employs a triple-differences approach that leverages variation across time, affiliation status, and whether the affiliation occurs within the same geographic market. This design isolates the incremental effect of affiliating with a hospital in the same market where competitive effects are most salient. Using this approach, I find that [INSERT FINDINGS].

    
    [IMPLICATIONS OF FINDINGS] Firms are responsive to antitrust regulation. That is, enforcement is a deterrant for anti-competitive behavior. Thus, it is important to think about whether regulation should increase even for small-scale affiliations. (something like this for policy implications?)


    Prior work establishes that corporations are deeply interconnected through overlapping directors and that these ties shape governance practices and firm performance (\cite{dooley1969interlocking}; \cite{ferris2003too}; \cite{field2013busy}; \cite{bizjak2009option}; \cite{bouwman2011corporate}; \cite{chiu2013board}; \cite{cai2014board}; \cite{baran2017director}; \cite{geng2021does}; \cite{schonlau2009board}; \cite{cai2012board}; \cite{cheng2021social}). That literature largely centers on publicly traded firms and frames overlap as either a governance friction (busy directors, option backdating, strategy convergence) or a source of private benefits (director networks that raise value, profitability, and post‑M\&A performance). In contrast, I study non‑merger affiliations in nonprofit hospitals, an arena often presumed less prone to anti-competitive behavior (\cite{baer2014clayton}; \cite{aai2013section7}) despite mounting evidence of competitive risks in health care (\cite{hulver2023ftc}). I broaden the construct beyond traditional board interlock to include executive cross‑board service, and I move the focus from governance proxies and firm‑level valuation to market‑relevant behaviors that are informed by the literature on both governance and consolidation (\cite{baran2017director}; \cite{geng2021does}; \cite{schonlau2009board}; \cite{cai2012board}; \cite{cheng2021social}). Methodologically, I leverage a triple‑differences design that distinguishes affiliations formed within versus across geographic markets to isolate the incremental, competition‑salient effect of overlap, rather than inferring behavior from financial outcomes alone. 

    A large empirical literature examines the effects of consolidation in health care, particularly through hospital mergers and system integration. Early work found little evidence that consolidation improved operating costs, efficiency, or quality of care (\cite{alexander1996short}; \cite{ho2000hospital}; \cite{dranove2003hospital}). Yet from 2000 to 2020, hospital mergers surged, leaving a quarter of markets with no independent hospitals by 2020 (\cite{ElevanceHealth2023}). This trend spurred extensive research on price, quality, and efficiency. Studies consistently document modest cost and efficiency gains for hospitals joining larger systems (\cite{schmitt2017hospital}; \cite{andreyeva2024corporatization}; \cite{craig2021mergers}), but these gains rarely translate into lower prices; instead, mergers typically lead to significant price increases as hospitals gain bargaining power with insurers (\cite{gaynor2012impact}; \cite{boozary2019association}; \cite{cooper2019price}; \cite{andreyeva2024corporatization}). Evidence on quality is mixed: most studies find little effect on readmissions or mortality (\cite{haas2011mergers}; \cite{beaulieu2020changes}), though some report increased treatment intensity and mortality for heart disease patients (\cite{hayford2012impact}) or higher readmission rates overall (\cite{andreyeva2024corporatization}). Beyond price and quality, research highlights mechanisms such as convergence in treatment styles (\cite{eliason2020acquisitions}), consolidated service offerings (\cite{mariani2022impact}), and redistribution of patients within markets, including declines in Medicaid admissions in highly concentrated areas (\cite{desai2023hospital}). While this literature focuses on mergers and system affiliation, my study examines a different form of structural connection: non-merger affiliations through board and executive overlap.

    The paper proceeds as follows. In Section \ref{sec:background}, I provide institutional details on nonprofit hospital governance and hospital affiliations. In Section \ref{sec:data}, I outline the data collection process and provide summary statistics. In Section \ref{sec:simpledid}, I estimate a simple staggered difference-in-differences model leveraging affiliations over time. Finally, in Section \ref{sec:tripledid}, I leverage variation in both affiliation and geography to estimate a triple differences specification. 


    

    


    \section{Background/Setting}\label{sec:background}

    \subsection{Nonprofit Hospital Governance}

    Nonprofits are governed at a high level by a board of directors: a group of unpaid community members that gather to oversee the broad direction and strategies of the firm. The board of directors does not mange the day-to-day activity of the hospital, but provides oversight to the highest level of manager, the Chief Executive Officer, whom the board also selects. The practical governance of boards is highly variable across different hospitals. State-level policies might influence details such as the regularity of board meetings or board member term limits. However, the main function of the board remains consistent: to provide strategic direction and oversight. In research focused on hospital governance, several factors have been found to be correlated with strategic change or hospital-level behaviors, including more influence over the CEO or top management (\cite{golden2001will}; \cite{alexander2008governance}; \cite{jiang2012enhancing}). Additionally, since the Affordable Care Act, boards have been actively involved in quality oversight (\cite{jha2010hospital}; \cite{prybil2014board}; \cite{prybil2010board}). Therefore, while boards are not usually involved in the day-to-day operations of hospitals, they do influence strategy and long-run hospital behavior.

    The board of directors selects a Chief Executive Officer, who then selects other members of the executive team such as a Chief Financial Officer and/or Chief Medical Officer. This leadership team manages the day-to-day operations of the hospital. An executive team usually consists of at least a Chief Executive Officer (CEO) and Chief Financial Officer (CFO), but there is variation in how firms organize these teams. Some hospital executives specialize in health care administration by earning a degree in health care management, or an MBA specific to health care. 

    In this paper, I consider the effects of board and executive interlock, someone sitting as an executive or on the board of multiple organizations. Board overlap does occur in health care with instances of organizations denied merger permission due to boards not acting independently (\cite{huberfeld2006tackling}), of hospitals declaring informal affiliations through leadership (\cite{barnett_babcock_2012}), and of life sciences companies with substantial documented overlap (\cite{manjunath2024illegal}). For many years, oversight of this activity in nonprofits was considered second-order to that of for-profits. However, in 2024, the FTC raised concern over board interlock of two prominent health care organizations (\cite{ftc2025sevita}). Additionally, board will sometimes ask executives from other organizations to sit on their board (\cite{gordon2025ftc}; \cite{alston2024governance}). 

    Early research on board interlock highlights that firms are not isolated entities but interconnected through shared directors (\cite{dooley1969interlocking}), with interlocks linked to firm size, financial ties, and management control. Research has examined governance implications, including whether overlapping boards overextend directors (\cite{ferris2003too}; \cite{field2013busy}), increase stock option backdating (\cite{bizjak2009option}), or promote convergence in governance strategies (\cite{bouwman2011corporate}; \cite{chiu2013board}; \cite{cai2014board}). Empirical evidence focuses largely on financial outcomes: well-connected directors can enhance firm value (\cite{baran2017director}), and board overlap is associated with higher profitability (\cite{geng2021does}), greater acquisition activity, and improved post-merger performance (\cite{schonlau2009board}; \cite{cai2012board}). Mechanisms include shared IT investments (\cite{cheng2021social}), reduced R\&D, and increased product differentiation (\cite{geng2021does}).


    \subsection{Hospital Affiliations}\label{sec:hospaffil}

    Board interlock is often happening in conjunction with affiliation agreements: any relationship between otherwise independent health care providers designed to create shared advantage and value (\cite{aha2018affiliations}). Some examples of affiliation agreements include joint ventures for service provision, clinical agreements among hired health care workers, or joint management. It is common for these types of agreements to involve some sort of shared governance (\cite{aha2018affiliations}; \cite{natlawreview2023ftc}). While some have raised anti-trust concerns over affiliation agreements, they are largely undocumented, and therefore their effects are unknown (\cite{rand2023consolidation}). Organizations do not require pre-authorization to form affiliation agreements, but there are several states that now require notice of any type of affiliation (\cite{ncsl2024healthcare}). 

    The more documented and studied type of hospital affiliation is that of traditional mergers and acquisitions. I will discuss the literature surrounding this topic in order to inform relevant behaviors that could also be affected by affiliation agreements. Hospital consolidation has accelerated over the past two decades, with mergers becoming so prevalent that by 2020, a quarter of hospital markets had no independent hospitals (\cite{ElevanceHealth2023}). Early studies found little evidence that consolidation improved operating costs, efficiency, or quality (\cite{alexander1996short}; \cite{ho2000hospital}; \cite{dranove2003hospital}), but more recent work documents modest efficiency gains for hospitals joining larger systems (\cite{schmitt2017hospital}; \cite{andreyeva2024corporatization}; \cite{craig2021mergers}). These gains, however, rarely translate into lower prices; instead, mergers consistently lead to significant price increases due to enhanced bargaining power with insurers (\cite{gaynor2012impact}; \cite{boozary2019association}; \cite{cooper2019price}; \cite{andreyeva2024corporatization}). Evidence on quality is mixed: most studies find little change in readmission or mortality rates (\cite{haas2011mergers}; \cite{beaulieu2020changes}), though some report higher treatment intensity and mortality for heart disease patients (\cite{hayford2012impact}) and increased readmissions overall (\cite{andreyeva2024corporatization}). Other research highlights mechanisms such as convergence in treatment styles (\cite{eliason2020acquisitions}), changes in service offerings (\cite{mariani2022impact}), and shifts in admission patterns, including reduced Medicaid admissions in concentrated markets (\cite{desai2023hospital}). These findings underscore that consolidation can influence not only costs and prices but also care delivery and patient access.

   
    \section{Data and Summary Statistics}\label{sec:data}

    
   To create a measure of board/executive interlock, I compile data on nonprofit hospital board of directors and executives in years 2017-2022 from publicly available Tax Form 990s. All sufficiently large nonprofits must file these forms with the Internal Revenue Service (IRS) each year. They contain a section in which nonprofits declare their board, executives, and highest compensated employees. This paper focuses only on the board of director identities. I limit to firms that fill out Schedule H of the tax form, an indicator that the firm operates a hospital. I extract all names and positions of individuals listed in the section titled ``Officers, Directors, Trustees, Key Employees, and Five Highest Compensated Employees". I also record information on the hospital that filed the tax form: the Employee Identification Number (EIN), organization name and location, and other characteristics. The initial extraction yields 2,096 nonprofit hospitals.

   Next, I match hospitals in the tax form data to hospitals in other publicly available data sources. The only common information about hospitals across these sources is the name and location. Therefore, I use fuzzy string matching methods to create a crosswalk of EINs to identification numbers in the American Hospital Association (AHA) survey data. Specifically, I compute the Jaro-Winkler distance between hospital names in each data set. Then, I record matches with sufficiently similar names and addresses, yielding a sample of 1,824 matched hospitals. 

    I record self-reported information about the individuals in the data, including whether they have a title for medical doctor, nurse, or having a degree in health administration. To clean the names, I remove any titles or credentials, and I convert common nicknames to their longer versions for consistency. I also combine names within the same firm that have either slight misspelling differences or, if middle names are only sometimes listed, they have at least 2 names in common. For example, if John Lee Matthew is a board member of hospital A in 2017, and John Matthew is a board member of hospital A in 2018, I assume these names represent the same person. Finally, I use the gender package in R to predict each individual's gender using historical data (\cite{gender}). 

    I present summary statistics of characteristics of the people in the data in Table \ref{tab:board_people}. There are just under 37,000 people in the data, 63\% of which are board members and the remaining 37\% are executives. Board members are slightly moe likely to be medical doctors than executives, but the fraction of doctor board members is still relatively low at 18\%. Nurses are consistent across board and executives, but only 3\% of the sample are nurses. A negligible number of people claim to have a health administration degree. Executives are slightly more likely to be female according to the name algorithms, with 38\% of executives estimated to be female and 32\% of board members estimated to be female. I also include average team composition of hospital boards and executive teams, where these trends largely hold at the hospital level. On average, hospital boards contain 7 members and executive teams contain 10 members. 

    \import{Objects}{boardandexec_people.tex}

    I create a list of overlapping board member candidates by identifying any names that occur on multiple hospital boards in the same year. However, it is very common for hospitals within a system to largely share the same board. This is not the type of variation I seek to investigate in this paper, since hospitals within systems are already heavily affiliated. Therefore, I define the relevant type of overlap as someone being on the board of multiple hospitals that are otherwise \textit{unaffiliated}. Hospitals within systems are still eligible to have overlap with other hospitals that do not belong to the same system. I exclude hospitals that gain or lose overlapping board members several times in the sample period. 
    
    In the hospital merger literature, mergers have been found to be more significant when the two hospitals involved are located geographically close (\cite{cooper2019price}). Therefore, I define two types of overlap: overlap with an otherwise non-affiliated hospital in the same market, or overlap with an otherwise unaffiliated hospital in a different market. I use the market definitions developed by Dartmouth Atlas, Hospital Referral Regions (HRRs), which are meant to capture geographic regions of the average patient's choice set of hospitals.


    \subsection{Documenting Leadership Overlap}

    In Figure \ref{fig:connected_percent} I show the prevalence of overlapping board members or executives over time. In 2017, approximately 5\% of hospitals have an overlapping board members or executive with another hospital in the same market that is not formally affiliated otherwise. That is, these hospital are not under common ownership. This percentage increases over time, with 10\% of hospitals sharing a key leader in their market in 2022. I also graph the percent of HRR markets where hospitals within that market share board members or executives. Approximately 15\% of HRRs are potentially affected by this type of affiliation. Sentence here about putting this into perspective relative to hospital consolidation as a whole. 
    
    \begin{figure}[ht!]
        \centering
        \vspace{9mm}
        \caption{Prevalence of Board/Executive Interlock}
        \includegraphics[width=.9\textwidth]{Objects/connected_percent.pdf}
        \label{fig:connected_percent}
    \end{figure}


    Next, I present the geographic distribution of these pairs in Figure \ref{fig:connected_maps}. The blue lines represent hospitals that share a common board member, and the red dots are all other hospitals. Connected hospitals are distributed fairly evenly across the US, apart from relatively few connections on the west coast relative to the population. Over time, the connections remain fairly consistent geographically. 

    \begin{figure}[ht!]
        \centering
        \caption{Geographic Distribution of Board/Executive Interlock}
        \includegraphics[width=\textwidth]{Objects/connected_maps.pdf}
        \label{fig:connected_maps}
    \end{figure}

    Hospitals in the sample are either always connected, never connected, or become connected at some point in time. I show characteristics of hospitals in these categories in Table \ref{tab:hosp_group_stats}. 

    \import{Objects}{hosp_group_stats.tex}

    I now present characteristics of pairs of hospitals that are connected by board members or executives in Table \ref{hospital_pair_types}. In the first column, I show pair characteristics of hospitals connected within the same HRR. In the second column, I present characteristics of hospitals connected to otherwise unaffiliated hospitals located in a different HRR. Therefore, all hospital pairs in the table are connected, but only the first column are connected to hospitals that are potentially direct competitors. Each hospital in a pair is general or specialty (I only consider adult hospitals for these categories), children or adult, and independent or belonging to a system, all characteristics in the AHA data. Additionally, the median bed size of all hospitals in the sample is 101, and I define hospitals as being either small or large depending on whether they fall below or above this median. There are typically 1.5 connections in the same HRR and almost 6 connections in different HRRs. 

    \import{Objects}{hospital_pair_types.tex}

    Most pairs, 76-78\% are two general hospitals. Thus, for the majority of hospitals that have board/executive interlock in the same HRR, the hospital they share a leader with is a direct competitor. I also include characteristics of system versus independent ownership. For there is a nearly even spread of the type of ownership of the pair. 42\% of these pairs are made up of one independent hospital and one system-owned hospital, while 37\% are made up of two independent hospitals and the rest are made up of two hospitals that are both system owned but belong to two different systems. These figures are slightly different for connections in different HRRs, where a larger portion of connections, 45\%, are made up of two system-owned hospitals of two different systems. 
    
    In both same and different HRR connected pairs, 50\% of the connections are such that one hospital is small in bed size and one is large in bed size. The average bed difference is over 200, indicating that the size difference of the average hospital connection is drastic. what's the implication of this? Finally, connected hospitals in the same HRR are approximately 62 kilometers, or 38 miles, away from the other.

    \import{Objects}{hosp_outcomes_table.tex}


    \section{Is Leadership Overlap Correlated with Consolidation?}

    While hospitals often publicly declare that their leadership affiliations are not intended to lead to mergers or other  consolidation (add citations), the relationship between leadership overlap and consolidation is an open empirical question.    Theoretically, there are three ways that leadership overlap and consolidation could be related. First, leadership overlap may act as a substitute for formal mergers. In this case, hospitals leverage shared governance to achieve some benefits of consolidation, such as strategic alignment, resource coordination, and cost efficiencies, without incurring the regulatory scrutiny and transaction costs associated with a merger. If this hypothesis holds, we would expect leadership overlap to reduce the likelihood of subsequent merger activity. Second, leadership overlap may serve as a precursor to merging. Hospitals might begin by sharing board members or executives to build trust, harmonize decision-making, and prepare for operational integration. Under this interpretation, overlap is an intermediate step toward full consolidation, and its presence should increase the probability of a future merger. Third, leadership overlap may result from intentions to merge. Here, overlap is not an independent strategic choice but rather a signal of an already-decided merger plan. This scenario introduces endogeneity concerns, as leadership overlap and merger intentions are jointly determined. In addition to its possible role in mergers, leadership overlap may also influence the likelihood of hospital closure. Shared leadership could provide access to broader networks, financial resources, and managerial expertise, which might help struggling hospitals remain viable and avoid closure. Conversely, overlap could signal strategic realignment within a system, where weaker hospitals are deliberately phased out to concentrate resources in stronger facilities. In this sense, leadership overlap could either reduce closure risk by stabilizing operations or increase it if consolidation strategies involve selective shutdowns.

    To evaluate these possibilities, I estimate the effect of gaining or losing leadership overlap on two outcomes: (1) the likelihood that a hospital closes in the future, and (2) the likelihood that it merges in the future. I employ a stacked difference-in-differences (DiD) approach, which constructs event-time observations for each treated hospital around the time of gaining or losing overlap and stacks these into a single panel. This method accommodates variation in treatment timing across hospitals and enables estimation of dynamic effects. 

    \begin{equation}
    Y_{it} = \alpha + \sum_{k \neq -1} \beta_k \cdot D_{i,t+k} + \gamma_i + \delta_t + \varepsilon_{it}
    \end{equation}

    \noindent where $Y_{it}$ is the outcome for hospital $i$ at time $t$ (closure or merger indicator), $D_{i,t+k}$ is an indicator for event time $k$ relative to the change in leadership overlap (with $k = -1$ as the omitted category), $\gamma_i$ and $\delta_t$ are hospital and time fixed effects, respectively, and $\beta_k$ captures the effect at each event time $k$.

    \begin{figure}[ht!]
        \centering
        \vspace{9mm}
        \caption{Effect of Overlap on Closure}
        \includegraphics[width=.9\textwidth]{Objects/closure.pdf}
        \label{fig:closure}
    \end{figure}

    \begin{figure}[ht!]
        \centering
        \vspace{9mm}
        \caption{Effect of Overlap on in-System Overlap}
        \includegraphics[width=.9\textwidth]{Objects/any_formal_sameHRR.pdf}
        \label{fig:any_formal_sameHRR}
    \end{figure}

    \begin{figure}[ht!]
        \centering
        \vspace{9mm}
        \caption{Effect of Overlap on Being Independent}
        \includegraphics[width=.9\textwidth]{Objects/independent.pdf}
        \label{fig:independent}
    \end{figure}


     


    
    \section{Conclusion}


    This study contributes to the growing literature on hospital governance and market structure by examining the extent and implications of board member overlap across hospitals. By identifying the prevalence of shared board membership, I provide new insights into how governance ties may influence competition, decision-making, and resource allocation in the hospital sector. This paper sheds light on the role of board interlocks in potentially shaping hospital behavior, offering an important perspective on the non-ownership mechanisms that can link hospitals within the same market.

    Using a comprehensive dataset of board members in nonprofit hospitals from 2017-2022, I document the characteristics of hospitals with shared board members and analyze how these overlaps have evolved over time. I identify connections between hospitals and examine the patterns of overlap by hospital type, market, and geographic region. The results indicate that a significant share of hospitals are connected through board interlocks, and a significant portion of hospital markets contain board overlap within the market. Further, I show geographic distributions of overlap and characterize connected hospital pairs by important characteristics such as ownership, size, and specialty. 

    These findings hold important implications for policymakers and future research. Regulators and antitrust authorities should consider the potential competitive effects of board member overlap when assessing hospital market dynamics, particularly in regions with high consolidation or limited provider choice. Future research could further investigate how these governance connections influence hospital pricing, quality of care, and patient outcomes. Expanding this work to explore causal relationships and the broader economic consequences of board interlocks would provide valuable insights for competition in health care.
  

    \newpage


    \printbibliography


    

    \appendix

\begin{figure}
    \centering
    \caption{Outcome Variables Over Time, Patient and Service Allocation}
    \includegraphics[width=0.75\linewidth]{Objects/outcome_perc_dis_bed_graph.pdf}
    \label{fig:outcome_means1}
\end{figure}

\begin{figure}
    \centering
    \caption{Outcome Variables Over Time, Purhcases, Costs, and Revenue}
    \includegraphics[width=0.75\linewidth]{Objects/outcome_purch_oper_rev_graph.pdf}
    \label{fig:outcome_means2}
\end{figure}

    

    

	
	
	


\end{document}