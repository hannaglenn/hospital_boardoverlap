\documentclass[12pt]{article}

\usepackage{tgtermes}
\usepackage{epsf}
\usepackage{epstopdf}
\usepackage{amsmath}
\usepackage{graphicx}
\usepackage{booktabs}
\usepackage[colorlinks=true,linkcolor=blue,citecolor=blue]{hyperref}
\usepackage{dcolumn}
\usepackage{amsmath, amsthm, amssymb}
\usepackage{mwe}
\usepackage{url}
%\usepackage{harvard}
\usepackage{fancyheadings}
\usepackage{longtable}
\usepackage{authblk}
\usepackage{setspace}
%\usepackage[nomarkers]{endfloat}
\usepackage{float}
\usepackage{bbm}
%\usepackage{titling}
\usepackage{subcaption}
\usepackage{algorithm}
\usepackage{algorithmic}
\usepackage{import}
\usepackage[backend=biber,style=authoryear,
sorting=nyt,citestyle=authoryear]{biblatex}
\addbibresource{papercitations.bib}
%\usepackage[nomarkers,nofiglist,notablist]{endfloat}
\usepackage{subcaption}
\usepackage{caption}

\onehalfspacing
\textwidth 6.5in \oddsidemargin 0in \evensidemargin -0.6in
\textheight 8.5in \topmargin -0.2in

\newcolumntype{L}[1]{>{\raggedright\let\newline\\
		\arraybackslash\hspace{0pt}}m{#1}}
\newcolumntype{C}[1]{>{\centering\let\newline\\
		\arraybackslash\hspace{0pt}}m{#1}}
\newcolumntype{R}[1]{>{\raggedleft\let\newline\\
		\arraybackslash\hspace{0pt}}m{#1}}
\newcolumntype{P}[1]{>{\raggedright\tabularxbackslash}p{#1}}

\newtheorem{theorem}{Theorem}[section]
\newtheorem{corollary}[theorem]{Corollary}
\newtheorem{proposition}[theorem]{Proposition}
\newtheorem{lemma}[theorem]{Lemma}

\captionsetup{justification=centering,singlelinecheck=false}


\newcommand{\xsub}[1]{%
	\mbox{\scriptsize\begin{tabular}{@{}c@{}}#1\end{tabular}}%
}

%\renewcommand{\thetable}{\Roman{table}}

\begin{document}
	
	
	
	
	\linespread{1.2}\title{\vspace{-0.5in} Unobserved Firm Affiliations: Evidence from Hospital Board Overlap} 
	
	\date{\today}
	
	\author{\vspace{10mm}Hanna Glenn\footnote{Department of Economics, Emory University, 1602 Fishburne Drive, Atlanta, GA 30322, hanna.glenn@emory.edu.} }
	
	\maketitle
	%\setlength{\droptitle}{-10pt}
	
	\vspace{-0.2in}
	
	\singlespacing\maketitle


 \vspace{3mm}
	
    \begin{abstract}
		{\small

		} 
	\end{abstract}
	
	
	
	
	

	
	\onehalfspacing
	
	\newpage

    \section{Introduction}

    Antitrust enforcement in the US has been particularly relevant in recent years as price gouging and consolidation has increased. 

    Antitrust legislation creates an additional hurdle for large-scale anti-competitive behavior, but there are other types of firm affiliations that typically elude antitrust scrutiny such as common ownership, joint ventures, and other types of affiliation/partnerships. By and large, firms form partnerships with other firms, often competitors, without needing to seek approval. This has started to raise concern recently, and states have even begun regulating smaller-scale affiliations in recent years. Horizontal mergers and consolidation have rightfully received a lot of attention in the literature, often linked to something?? However, we know little about the effects of smaller-scale affiliations, which could carry some of the same implications as consolidation proposals scrutinized for their ant-competitive nature. 

    In this paper, I examine the effects of soft consolidation in the form of overlapping board members in nonprofit hospitals. Existing research establishes a strong link between board overlap and governance practices, but we know little about whether there should be anti-competitive concerns when firms share common board members. Additionally, most studies have been limited to large, publicly traded firms. This focus stems from the assumption that nonprofits engage in less anti-competitive behavior (\cite{baer2014clayton}; \cite{aai2013section7}), making board overlap in such organizations less concerning. However, this assumption is flawed, particularly in industries like health care, where nonprofit hospitals dominate and have been shown to engage in anti-competitive practices (\cite{hulver2023ftc}). Despite these insights, gaps persist: research rarely addresses market-level effects or nonprofit sectors, assuming nonprofits pose minimal antitrust risk (\cite{baer2014clayton}; \cite{aai2013section7}). This assumption is problematic in health care, where nonprofit hospitals dominate and exhibit anti-competitive behavior (\cite{hulver2023ftc}). Further work should explore competitive implications of board overlap beyond governance and financial metrics, particularly in nonprofit contexts.

    In 2024, the FTC banned a case of interlocking board (\cite{ftc2025sevita}). 
    


    \section{Background/Setting}

    \subsection{Nonprofit Hospital Governance}

    Nonprofits are governed at a high level by a board of directors: a group of unpaid community members that gather to oversee the broad direction and strategies of the firm. The board of directors does not mange the day-to-day activity of the hospital, but provides oversight to the highest level of manager, the Chief Executive Officer, whom the board also selects. The practical governance of boards is highly variable across different hospitals. State-level policies might influence details such as the regularity of board meetings or board member term limits. However, the main function of the board remains consistent: to provide strategic direction and oversight. In research focused on hospital governance, several factors have been found to be correlated with strategic change or hospital-level behaviors, including more influence over the CEO or top management (\cite{golden2001will}; \cite{alexander2008governance}; \cite{jiang2012enhancing}). Additionally, since the Affordable Care Act, boards have been actively involved in quality oversight (\cite{jha2010hospital}; \cite{prybil2014board}; \cite{prybil2010board}). Therefore, while boards are not usually involved in the day-to-day operations of hospitals, they do influence strategy and long-run hospital behavior.

    The board of directors selects a Chief Executive Officer, who then selects other members of the executive team such as a Chief Financial Officer and/or Chief Medical Officer. This leadership team manages the day-to-day operations of the hospital. An executive team usually consists of at least a Chief Executive Officer (CEO) and Chief Financial Officer (CFO), but there is variation in how firms organize these teams. Some hospital executives specialize in health care administration by earning a degree in health care management, or an MBA specific to health care. 

    In this paper, I consider the effects of board and executive interlock, someone sitting as an executive or on the board of multiple organizations. Board overlap does occur in health care with instances of organizations denied merger permission due to boards not acting independently (\cite{huberfeld2006tackling}), of hospitals declaring informal affiliations through leadership (\cite{barnett_babcock_2012}), and of life sciences companies with substantial documented overlap (\cite{manjunath2024illegal}). For many years, oversight of this activity in nonprofits was considered second-order to that of for-profits. However, in 2024, the FTC raised concern over board interlock of two prominent health care organizations (\cite{ftc2025sevita}). Additionally, board will sometimes ask executives from other organizations to sit on their board (\cite{gordon2025ftc}; \cite{alston2024governance}). 

    Early research on board interlock highlights that firms are not isolated entities but interconnected through shared directors (\cite{dooley1969interlocking}), with interlocks linked to firm size, financial ties, and management control. Research has examined governance implications, including whether overlapping boards overextend directors (\cite{ferris2003too}; \cite{field2013busy}), increase stock option backdating (\cite{bizjak2009option}), or promote convergence in governance strategies (\cite{bouwman2011corporate}; \cite{chiu2013board}; \cite{cai2014board}). Empirical evidence focuses largely on financial outcomes: well-connected directors can enhance firm value (\cite{baran2017director}), and board overlap is associated with higher profitability (\cite{geng2021does}), greater acquisition activity, and improved post-merger performance (\cite{schonlau2009board}; \cite{cai2012board}). Mechanisms include shared IT investments (\cite{cheng2021social}), reduced R\&D, and increased product differentiation (\cite{geng2021does}).


    \subsection{Hospital Affiliations}\label{sec:hospaffil}

    Board interlock is often happening in conjunction with affiliation agreements: any relationship between otherwise independent health care providers designed to create shared advantage and value (\cite{aha2018affiliations}). Some examples of affiliation agreements include joint ventures for service provision, clinical agreements among hired health care workers, or joint management. It is common for these types of agreements to involve some sort of shared governance (\cite{aha2018affiliations}; \cite{natlawreview2023ftc}). While some have raised anti-trust concerns over affiliation agreements, they are largely undocumented, and therefore their effects are unknown (\cite{rand2023consolidation}). Organizations do not require pre-authorization to form affiliation agreements, but there are several states that now require notice of any type of affiliation (\cite{ncsl2024healthcare}). 

    The more documented and studied type of hospital affiliation is that of traditional mergers and acquisitions. I will discuss the literature surrounding this topic in order to inform relevant behaviors that could also be affected by affiliation agreements. Hospital consolidation has accelerated over the past two decades, with mergers becoming so prevalent that by 2020, a quarter of hospital markets had no independent hospitals (\cite{ElevanceHealth2023}). Early studies found little evidence that consolidation improved operating costs, efficiency, or quality (\cite{alexander1996short}; \cite{ho2000hospital}; \cite{dranove2003hospital}), but more recent work documents modest efficiency gains for hospitals joining larger systems (\cite{schmitt2017hospital}; \cite{andreyeva2024corporatization}; \cite{craig2021mergers}). These gains, however, rarely translate into lower prices; instead, mergers consistently lead to significant price increases due to enhanced bargaining power with insurers (\cite{gaynor2012impact}; \cite{boozary2019association}; \cite{cooper2019price}; \cite{andreyeva2024corporatization}). Evidence on quality is mixed: most studies find little change in readmission or mortality rates (\cite{haas2011mergers}; \cite{beaulieu2020changes}), though some report higher treatment intensity and mortality for heart disease patients (\cite{hayford2012impact}) and increased readmissions overall (\cite{andreyeva2024corporatization}). Other research highlights mechanisms such as convergence in treatment styles (\cite{eliason2020acquisitions}), changes in service offerings (\cite{mariani2022impact}), and shifts in admission patterns, including reduced Medicaid admissions in concentrated markets (\cite{desai2023hospital}). These findings underscore that consolidation can influence not only costs and prices but also care delivery and patient access.

   
    \section{Data Construction}

    
   To create a measure of board/executive interlock, I compile data on nonprofit hospital board of directors and executives in years 2017-2022 from publicly available Tax Form 990s. All sufficiently large nonprofits must file these forms with the Internal Revenue Service (IRS) each year. They contain a section in which nonprofits declare their board, executives, and highest compensated employees. This paper focuses only on the board of director identities. I limit to firms that fill out Schedule H of the tax form, an indicator that the firm operates a hospital. I extract all names and positions of individuals listed in the section titled ``Officers, Directors, Trustees, Key Employees, and Five Highest Compensated Employees". I also record information on the hospital that filed the tax form: the Employee Identification Number (EIN), organization name and location, and other characteristics. The initial extraction yields 2,096 nonprofit hospitals.

   Next, I match hospitals in the tax form data to hospitals in other publicly available data sources. The only common information about hospitals across these sources is the name and location. Therefore, I use fuzzy string matching methods to create a crosswalk of EINs to identification numbers in the American Hospital Association (AHA) survey data. Specifically, I compute the Jaro-Winkler distance between hospital names in each data set. Then, I record matches with sufficiently similar names and addresses, yielding a sample of 1,824 matched hospitals. 

     I record self-reported information about the individuals in the data, including whether they have a title for medical doctor, nurse, or having a degree in health administration. To clean the names, I remove any titles or credentials, and I convert common nicknames to their longer versions for consistency. I also combine names within the same firm that have either slight misspelling differences or, if middle names are only sometimes listed, they have at least 2 names in common. For example, if John Lee Matthew is a board member of hospital A in 2017, and John Matthew is a board member of hospital A in 2018, I assume these names represent the same person. Finally, I use the gender package in R to predict each individual's gender using historical data (\cite{gender}). 

    \import{Objects}{boardand_people.tex}

    I create a list of overlapping board member candidates by identifying any names that occur on multiple hospital boards in the same year. However, it is very common for hospitals within a system to largely share the same board. This is not the type of variation I seek to investigate in this paper, since hospitals within systems are already heavily affiliated. Therefore, I define the relevant type of overlap as someone being on the board of multiple hospitals that are otherwise \textit{unaffiliated}. Hospitals within systems are still eligible to have overlap with other hospitals that do not belong to the same system as they do. Further, In the hospital merger literature, mergers have been found to be more significant when the two hospitals involved are located geographically close (\cite{cooper2019price}). Therefore, I define two types of overlap: overlap with an otherwise non-affiliated hospital in the same market, or overlap with an otherwise unaffiliated hospital in a different market. All other hospitals have no board overlap with otherwise unaffiliated hospitals. I use the market definitions developed by Dartmouth Atlas, Hospital Referral Regions (HRRs), which are meant to capture geographic regions of the average patient's choice set of hospitals.

    The final sample limitation I make for simplicity is to remove hospitals that gain and then lose overlapping board members in the sample. Since treatment will be turning on overlapping board member, I remove those who "lose" their treatment. 

    \subsection{Documenting Overlapping Board Members}

    The full sample of hospitals 



    \subsection{Summary Statistics}

    First, I present a summary statistics table showing characteristics of all hospitals in the final sample, which consists of 1,467 nonprofit hospitals, almost all of which are adult general hospitals. They have, on average, 178 beds and 14 board members. Almost 40\% of the hospitals in the sample are academic medical centers, almost 70\% of the sample are located in metropolitan areas, and 56\% belong to a system. 

    %\import{Objects}{all_summarystats.tex}

    Eight percent of hospitals in the sample have overlapping board members with an otherwise unaffiliated hospital. In Figure \ref{fig:connected_percent}, I show that this percentage remains relatively constant over time, with a slight decline from 129 affiliated hospitals in 2017 to 114 affiliated hospitals in 2021. I include 2022 values, but there are less total firms in the sample due to lag in tax forms being made available publicly.
    
    \begin{figure}[ht!]
        \centering
        \vspace{9mm}
        \caption{Percent of hospitals that have a board affiliation}
        \includegraphics[width=.8\textwidth]{Objects/connected_percent.pdf}
        \label{fig:connected_percent}
    \end{figure}
    
    As mentioned previously, common board members should be more likely to affect behavior within a hospital market, where hospitals compete for the same patient population. In Figure \ref{fig:connected_HRR_percent}, I show the prevalence of having shared board members within a hospital market. Of the 306 HRRs in the US, approximately 17\% have at least one pair of board affiliated hospitals within the market. This proportion decreases slightly over time, with 15\% of HRRs having connected hospitals in 2021, a nominal decrease of 8 markets losing hospitals with affiliation. This is likely due to increased system affiliations, which mechanically decreases board overlap since I do not include hospitals in the same system as eligible for having board overlap. 


    \begin{figure}[ht!]
        \centering
        \caption{Percent of HRRs containing board affiliated hospitals}
        \includegraphics[width=.8\textwidth]{Objects/connected_HRR_percent.pdf}
        \label{fig:connected_HRR_percent}
    \end{figure}

    Next, I present the geographic distribution of these pairs in Figure \ref{fig:connected_maps}. The blue lines represent hospitals that share a common board member, and the red dots are all other hospitals. Connected hospitals are distributed fairly evenly across the US, apart from relatively few connections on the west coast relative to the population. Over time, the connections remain fairly consistent geographically. Taken together, these show that board overlap is a prevalent occurrence among nonprofit hospitals in the US from 2017 onward. 

    \begin{figure}[ht!]
        \centering
        \caption{Geographic distribution of board affiliated hospitals over time}
        \includegraphics[width=\textwidth]{Objects/connected_maps.pdf}
        \label{fig:connected_maps}
    \end{figure}



    I now shift to examining characteristics of hospitals that share board members. I consider pairs of connected hospitals and characteristics of the pair in Table \ref{tab:hospital_pair_types}. Each hospital in a pair is general or specialty (I only consider adult hospitals for these categories), children or adult, and independent or belonging to a system, all characteristics in the AHA data. Additionally, the median bed size of all hospitals in the sample is 103, and I define hospitals as being either small or large depending on whether they fall below or above this median. 

    \import{Objects}{hospital_pair_types.tex}

    Two-thirds of board affiliated pairs are between two adult general hospitals, and 10\% of pairs are between a specialty and a general hospital. Thus, for the majority of hospitals that have board interlock, the hospital they share a board member with is a direct competitor. Additionally, there is significant variation in the ownership type of hospitals within affiliated pairs. Forty-five percent of pairs are between two non-system affiliated hospitals, and 36\% of pairs are between an independent hospital affiliated with a hospital belonging to a system. The lowest proportion of shared board member pairs is made up of two hospitals belonging to different systems. 
    
    There is also variation in whether the hospitals are the same size or not. Half of the hospital pairs consist of one small and one large hospital, and the other half are split evenly between two small or two large hospitals. However, the average difference in the number of beds of the two hospitals in a pair is very large: an average difference of 254 beds. This suggests that for hospitals of opposite size, the difference is drastic. The average distance between hospitals that share common board members is 35 miles, or 57 kilometers. In total, there are 217 hospitals that share a board member with at least one other hospital at some point in the sample. 

    \subsection{Correlations with Hospital Behaviors}

    I now merge the hospital board data to other publicly available data sets to examine descriptive differences in the hospital behaviors discussed in Section \ref{sec:hospbehaviors} for hospitals with and without board overlap. The broad categories of hospital behaviors that might be affected by affiliation through shared board members are price/quality, investment decisions, and differentiation, which are discussed in Section \ref{sec:hospbehaviors}. While I do not capture every aspect of hospital behavior, I select a number of variables that fall into these categories to examine. 

    First, from the Center for Medicare and Medicaid Services (CMS) Hospital Compare data, I merge information on each hospital's overall star rating. This rating combines measures of quality into one index that ranges from 1 to 5: readmission rates, mortality rates, safety, timeliness of care, effectiveness of care, and patient experience. This is a very aggregated measure, but may capture big-picture differences in quality.

    Next, I include data on different types of investments made into the hospital from the Healthcare Cost Reporting Information System (HCRIS) data. These variables include the dollar amounts spent on health information technology purchases, fixed equipment, and movable equipment. 

    There are a plethora of ways one could measure differentiation in hospitals. I focus on two categories of differentiation: service offerings and patient population. To measure different service offerings, I first construct a measure of concentration of beds devoted to categories of services in the AHA survey. For services $j$ in the list of all possible services\footnote{General medical and surgical (adult), General medical and surgical (pediatric), Obstetric care, Medical/surgical intensive care, Cardiac intensive care, Neonatal intensive care, Neonatal intermediate care, Pediatric intensive care beds, Burn care, Other Special Care, Other intensive care, Physical Rehabilitation care, Alcohol/drug abuse or dependency inpatient care, Psychiatric care, Skilled nursing care, Intermediate nursing care, Acute long term care, Other long-term care, Other care}, the concentration of bed allocation is defined as 

    $$concentration_{bed} = \sum_{j}\left(\frac{beds_j}{total beds}\right)^2.$$

    \noindent That is, a higher bed concentration indicates more specialization in bed allocation, where a lower concentration indicates more services spread out in bed allocations. Additionally, I include a second concentration measure that relies less on the facility and more on the conditions patients are treated with. The CMS Provider Utilization files contain measures of the percent of patients seen for different conditions. Therefore, for services $k$ in the list of conditions in the CMS data\footnote{Percent Of Beneficiaries Identified With Combined Cancer Flag For 6 Cancer Indicators, Percent Of Beneficiaries Identified With Chronic Obstructive Pulmonary Disease, Percent Of Beneficiaries Identified With Chronic Kidney Disease, Percent Of Beneficiaries Identified With Heart Failure And Non-Ischemic Heart Disease}, I define the concentration of patient conditions in each hospital as 

    $$HHI_{patient} = \sum_{k}\left(\frac{patients_k}{total patients}\right)^2.$$
    
    Finally, I include indicators for whether the hospital has a NICU or Cath Lab, both profitable services for hospitals to offer. I also consider differentiation by the patient population admitted to the hospital. There is information in the HCRIS data about how many discharges are for patients on Medicaid or Medicare. There is also information in the CMS Provider Utilization Files about the average risk score of Medicare patients seen in the hospital. I include all of these variables to capture something about the aggregate patient population being treated in different hospitals.

    In Table \ref{size_characteristics}, I present averages of all of these variables for four categories of hospitals: whether they are small (less than 103 beds) or large (greater than 103 beds), and whether they share board members with another competing hospital. I limit this sample to only general hospitals for relevant comparison. 

   \import{Objects}{size_characteristics}

   Small hospitals with and without shared board members have a very similar average star rating of 3.4, while large hospitals with shared board members have a slightly smaller average star rating than large hospitals without board overlap. In all categories of investment, small hospitals are relatively similar, with slightly more investment in health IT and fixed equipment for board affiliated small hospitals. Large hospitals are also generally similar in how they invest in health IT and fixed equipment relative to how much they spend on average, but hospitals with shared board members spend significantly more on movable equipment than hospitals without board overlap. 

   There do not seem to be meaningful differences in the concentration of services offered or conditions treated among hospitals with and without board overlap, whether they are small or large. Similarly, there are no clear differences in the patient composition of hospitals with and without board overlap. This is consistent even among the facilities of these hospitals, with no distinct differences in whether hospitals have a NICU or Cath lab based on board overlap. 

   Another important level of variation in this sample of hospitals could be whether they belong to a system or not, as this often dictates a number of the behaviors discussed. Therefore, I present a similar table of means for hospitals with and without board overlap that do or do not have a system affiliation in Table \ref{sys_characteristics}. 

   \import{Objects}{system_characteristics}

    Among hospitals that do not belong to a system, those without board overlap have a slightly higher quality rating than those without board overlap. However, the opposite is true among hospitals that do belong to a system, where those with board overlap have a higher quality rating. The most significant difference in investment is among hospitals that belong to a system for investment in movable equipment. Those with board overlap invest over 20 million in movable equipment, while those without board overlap only invest 8.3 million. The other comparisons of investment show no drastic differences. In-system and out-of-system hospitals both show higher levels of concentration of services when they do not have board overlap. However, the patient composition is similar within the categories of system affiliation. Independent hospitals that do not have board overlap are more likely to have a Cath lab, but otherwise hospitals are similar in in service offerings. 

    While these statistics do not establish a causal link between board overlap and hospital behavior, they highlight key differences across hospital categories. Notably, hospitals with overlapping board members exhibit distinct patterns in quality and investment decisions. These differences underscore the need for further research to explore board overlap as a potential mechanism for affiliation and behavioral changes in health care.
    
    \section{Conclusion}


    This study contributes to the growing literature on hospital governance and market structure by examining the extent and implications of board member overlap across hospitals. By identifying the prevalence of shared board membership, I provide new insights into how governance ties may influence competition, decision-making, and resource allocation in the hospital sector. This paper sheds light on the role of board interlocks in potentially shaping hospital behavior, offering an important perspective on the non-ownership mechanisms that can link hospitals within the same market.

    Using a comprehensive dataset of board members in nonprofit hospitals from 2017-2022, I document the characteristics of hospitals with shared board members and analyze how these overlaps have evolved over time. I identify connections between hospitals and examine the patterns of overlap by hospital type, market, and geographic region. The results indicate that a significant share of hospitals are connected through board interlocks, and a significant portion of hospital markets contain board overlap within the market. Further, I show geographic distributions of overlap and characterize connected hospital pairs by important characteristics such as ownership, size, and specialty. 

    These findings hold important implications for policymakers and future research. Regulators and antitrust authorities should consider the potential competitive effects of board member overlap when assessing hospital market dynamics, particularly in regions with high consolidation or limited provider choice. Future research could further investigate how these governance connections influence hospital pricing, quality of care, and patient outcomes. Expanding this work to explore causal relationships and the broader economic consequences of board interlocks would provide valuable insights for competition in health care.
  

    \newpage


    \printbibliography


    

    

    

    

    

	
	
	


\end{document}