\documentclass[12pt]{article}

\usepackage{tgtermes}
\usepackage{epsf}
\usepackage{epstopdf}
\usepackage{amsmath}
\usepackage{graphicx}
\usepackage{booktabs}
\usepackage[colorlinks=true,linkcolor=blue,citecolor=blue]{hyperref}
\usepackage{dcolumn}
\usepackage{amsmath, amsthm, amssymb}
\usepackage{mwe}
\usepackage{url}
%\usepackage{harvard}
\usepackage{fancyheadings}
\usepackage{longtable}
\usepackage{authblk}
\usepackage{setspace}
%\usepackage[nomarkers]{endfloat}
\usepackage{float}
\usepackage{bbm}
%\usepackage{titling}
\usepackage{subcaption}
\usepackage{algorithm}
\usepackage{algorithmic}
\usepackage{import}
\usepackage[backend=biber,style=authoryear,
sorting=nyt,citestyle=authoryear]{biblatex}
\addbibresource{papercitations.bib}
%\usepackage[nomarkers,nofiglist,notablist]{endfloat}
\usepackage{subcaption}
\usepackage{caption}

\onehalfspacing
\textwidth 6.5in \oddsidemargin 0in \evensidemargin -0.6in
\textheight 8.5in \topmargin -0.2in

\newcolumntype{L}[1]{>{\raggedright\let\newline\\
		\arraybackslash\hspace{0pt}}m{#1}}
\newcolumntype{C}[1]{>{\centering\let\newline\\
		\arraybackslash\hspace{0pt}}m{#1}}
\newcolumntype{R}[1]{>{\raggedleft\let\newline\\
		\arraybackslash\hspace{0pt}}m{#1}}
\newcolumntype{P}[1]{>{\raggedright\tabularxbackslash}p{#1}}

\newtheorem{theorem}{Theorem}[section]
\newtheorem{corollary}[theorem]{Corollary}
\newtheorem{proposition}[theorem]{Proposition}
\newtheorem{lemma}[theorem]{Lemma}

\captionsetup{justification=centering,singlelinecheck=false}


\newcommand{\xsub}[1]{%
	\mbox{\scriptsize\begin{tabular}{@{}c@{}}#1\end{tabular}}%
}

%\renewcommand{\thetable}{\Roman{table}}

\begin{document}
	
	
	
	
	\linespread{1.2}\title{\vspace{-0.5in} Unobserved Firm Affiliations: Evidence from Hospital Board Overlap} 
	
	\date{\today}
	
	\author{\vspace{10mm}Hanna Glenn\footnote{Department of Economics, Emory University, 1602 Fishburne Drive, Atlanta, GA 30322, hanna.glenn@emory.edu.} }
	
	\maketitle
	%\setlength{\droptitle}{-10pt}
	
	\vspace{-0.2in}
	
	\singlespacing\maketitle


 \vspace{3mm}
	
    \begin{abstract}
		{\small

		} 
	\end{abstract}
	
	
	
	
	

	
	\onehalfspacing
	
	\newpage

    \section{Introduction}

    Antitrust enforcement in the US has been particularly relevant in recent years as price gouging and consolidation has increased. 

    Antitrust legislation creates an additional hurdle for large-scale anti-competitive behavior, but there are other types of firm affiliations that typically elude antitrust scrutiny such as common ownership, joint ventures, and other types of affiliation/partnerships. By and large, firms form partnerships with other firms, often competitors, without needing to seek approval. This has started to raise concern recently, and states have even begun regulating smaller-scale affiliations in recent years. Horizontal mergers and consolidation have rightfully received a lot of attention in the literature, often linked to something?? However, we know little about the effects of smaller-scale affiliations, which could carry some of the same implications as consolidation proposals scrutinized for their ant-competitive nature. 

    In this paper, I examine the effects of soft consolidation in the form of overlapping board members in nonprofit hospitals. Existing research establishes a strong link between board overlap and governance practices, but we know little about whether there should be anti-competitive concerns when firms share common board members. Additionally, most studies have been limited to large, publicly traded firms. This focus stems from the assumption that nonprofits engage in less anti-competitive behavior (\cite{baer2014clayton}; \cite{aai2013section7}), making board overlap in such organizations less concerning. However, this assumption is flawed, particularly in industries like health care, where nonprofit hospitals dominate and have been shown to engage in anti-competitive practices (\cite{hulver2023ftc}). Despite these insights, gaps persist: research rarely addresses market-level effects or nonprofit sectors, assuming nonprofits pose minimal antitrust risk (\cite{baer2014clayton}; \cite{aai2013section7}). This assumption is problematic in health care, where nonprofit hospitals dominate and exhibit anti-competitive behavior (\cite{hulver2023ftc}). Further work should explore competitive implications of board overlap beyond governance and financial metrics, particularly in nonprofit contexts.

    In 2024, the FTC banned a case of interlocking board (\cite{ftc2025sevita}). 
    


    \section{Background/Setting}

    \subsection{Nonprofit Hospital Governance}

    Nonprofits are governed at a high level by a board of directors: a group of unpaid community members that gather to oversee the broad direction and strategies of the firm. The board of directors does not mange the day-to-day activity of the hospital, but provides oversight to the highest level of manager, the Chief Executive Officer, whom the board also selects. The practical governance of boards is highly variable across different hospitals. State-level policies might influence details such as the regularity of board meetings or board member term limits. However, the main function of the board remains consistent: to provide strategic direction and oversight. In research focused on hospital governance, several factors have been found to be correlated with strategic change or hospital-level behaviors, including more influence over the CEO or top management (\cite{golden2001will}; \cite{alexander2008governance}; \cite{jiang2012enhancing}). Additionally, since the Affordable Care Act, boards have been actively involved in quality oversight (\cite{jha2010hospital}; \cite{prybil2014board}; \cite{prybil2010board}). Therefore, while boards are not usually involved in the day-to-day operations of hospitals, they do influence strategy and long-run hospital behavior.

    The board of directors selects a Chief Executive Officer, who then selects other members of the executive team such as a Chief Financial Officer and/or Chief Medical Officer. This leadership team manages the day-to-day operations of the hospital. An executive team usually consists of at least a Chief Executive Officer (CEO) and Chief Financial Officer (CFO), but there is variation in how firms organize these teams. Some hospital executives specialize in health care administration by earning a degree in health care management, or an MBA specific to health care. 

    In this paper, I consider the effects of board and executive interlock, someone sitting as an executive or on the board of multiple organizations. Board overlap does occur in health care with instances of organizations denied merger permission due to boards not acting independently (\cite{huberfeld2006tackling}), of hospitals declaring informal affiliations through leadership (\cite{barnett_babcock_2012}), and of life sciences companies with substantial documented overlap (\cite{manjunath2024illegal}). For many years, oversight of this activity in nonprofits was considered second-order to that of for-profits. However, in 2024, the FTC raised concern over board interlock of two prominent health care organizations (\cite{ftc2025sevita}). Additionally, board will sometimes ask executives from other organizations to sit on their board (\cite{gordon2025ftc}; \cite{alston2024governance}). 

    Early research on board interlock highlights that firms are not isolated entities but interconnected through shared directors (\cite{dooley1969interlocking}), with interlocks linked to firm size, financial ties, and management control. Research has examined governance implications, including whether overlapping boards overextend directors (\cite{ferris2003too}; \cite{field2013busy}), increase stock option backdating (\cite{bizjak2009option}), or promote convergence in governance strategies (\cite{bouwman2011corporate}; \cite{chiu2013board}; \cite{cai2014board}). Empirical evidence focuses largely on financial outcomes: well-connected directors can enhance firm value (\cite{baran2017director}), and board overlap is associated with higher profitability (\cite{geng2021does}), greater acquisition activity, and improved post-merger performance (\cite{schonlau2009board}; \cite{cai2012board}). Mechanisms include shared IT investments (\cite{cheng2021social}), reduced R\&D, and increased product differentiation (\cite{geng2021does}).


    \subsection{Hospital Affiliations}\label{sec:hospaffil}

    Board interlock is often happening in conjunction with affiliation agreements: any relationship between otherwise independent health care providers designed to create shared advantage and value (\cite{aha2018affiliations}). Some examples of affiliation agreements include joint ventures for service provision, clinical agreements among hired health care workers, or joint management. It is common for these types of agreements to involve some sort of shared governance (\cite{aha2018affiliations}; \cite{natlawreview2023ftc}). While some have raised anti-trust concerns over affiliation agreements, they are largely undocumented, and therefore their effects are unknown (\cite{rand2023consolidation}). Organizations do not require pre-authorization to form affiliation agreements, but there are several states that now require notice of any type of affiliation (\cite{ncsl2024healthcare}). 

    The more documented and studied type of hospital affiliation is that of traditional mergers and acquisitions. I will discuss the literature surrounding this topic in order to inform relevant behaviors that could also be affected by affiliation agreements. Hospital consolidation has accelerated over the past two decades, with mergers becoming so prevalent that by 2020, a quarter of hospital markets had no independent hospitals (\cite{ElevanceHealth2023}). Early studies found little evidence that consolidation improved operating costs, efficiency, or quality (\cite{alexander1996short}; \cite{ho2000hospital}; \cite{dranove2003hospital}), but more recent work documents modest efficiency gains for hospitals joining larger systems (\cite{schmitt2017hospital}; \cite{andreyeva2024corporatization}; \cite{craig2021mergers}). These gains, however, rarely translate into lower prices; instead, mergers consistently lead to significant price increases due to enhanced bargaining power with insurers (\cite{gaynor2012impact}; \cite{boozary2019association}; \cite{cooper2019price}; \cite{andreyeva2024corporatization}). Evidence on quality is mixed: most studies find little change in readmission or mortality rates (\cite{haas2011mergers}; \cite{beaulieu2020changes}), though some report higher treatment intensity and mortality for heart disease patients (\cite{hayford2012impact}) and increased readmissions overall (\cite{andreyeva2024corporatization}). Other research highlights mechanisms such as convergence in treatment styles (\cite{eliason2020acquisitions}), changes in service offerings (\cite{mariani2022impact}), and shifts in admission patterns, including reduced Medicaid admissions in concentrated markets (\cite{desai2023hospital}). These findings underscore that consolidation can influence not only costs and prices but also care delivery and patient access.

   
    \section{Data and Summary Statistics}

    
   To create a measure of board/executive interlock, I compile data on nonprofit hospital board of directors and executives in years 2017-2022 from publicly available Tax Form 990s. All sufficiently large nonprofits must file these forms with the Internal Revenue Service (IRS) each year. They contain a section in which nonprofits declare their board, executives, and highest compensated employees. This paper focuses only on the board of director identities. I limit to firms that fill out Schedule H of the tax form, an indicator that the firm operates a hospital. I extract all names and positions of individuals listed in the section titled ``Officers, Directors, Trustees, Key Employees, and Five Highest Compensated Employees". I also record information on the hospital that filed the tax form: the Employee Identification Number (EIN), organization name and location, and other characteristics. The initial extraction yields 2,096 nonprofit hospitals.

   Next, I match hospitals in the tax form data to hospitals in other publicly available data sources. The only common information about hospitals across these sources is the name and location. Therefore, I use fuzzy string matching methods to create a crosswalk of EINs to identification numbers in the American Hospital Association (AHA) survey data. Specifically, I compute the Jaro-Winkler distance between hospital names in each data set. Then, I record matches with sufficiently similar names and addresses, yielding a sample of 1,824 matched hospitals. 

    I record self-reported information about the individuals in the data, including whether they have a title for medical doctor, nurse, or having a degree in health administration. To clean the names, I remove any titles or credentials, and I convert common nicknames to their longer versions for consistency. I also combine names within the same firm that have either slight misspelling differences or, if middle names are only sometimes listed, they have at least 2 names in common. For example, if John Lee Matthew is a board member of hospital A in 2017, and John Matthew is a board member of hospital A in 2018, I assume these names represent the same person. Finally, I use the gender package in R to predict each individual's gender using historical data (\cite{gender}). 

    I present summary statistics of characteristics of the people in the data in Table \ref{tab:board_people}. There are just under 37,000 people in the data, 63\% of which are board members and the remaining 37\% are executives. Board members are slightly moe likely to be medical doctors than executives, but the fraction of doctor board members is still relatively low at 18\%. Nurses are consistent across board and executives, but only 3\% of the sample are nurses. A negligible number of people claim to have a health administration degree. Executives are slightly more likely to be female according to the name algorithms, with 38\% of executives estimated to be female and 32\% of board members estimated to be female. I also include average team composition of hospital boards and executive teams, where these trends largely hold at the hospital level. On average, hospital boards contain 7 members and executive teams contain 10 members. 

    \import{Objects}{boardandexec_people.tex}

    I create a list of overlapping board member candidates by identifying any names that occur on multiple hospital boards in the same year. However, it is very common for hospitals within a system to largely share the same board. This is not the type of variation I seek to investigate in this paper, since hospitals within systems are already heavily affiliated. Therefore, I define the relevant type of overlap as someone being on the board of multiple hospitals that are otherwise \textit{unaffiliated}. Hospitals within systems are still eligible to have overlap with other hospitals that do not belong to the same system. I exclude hospitals that gain or lose overlapping board members several times in the sample period. 
    
    In the hospital merger literature, mergers have been found to be more significant when the two hospitals involved are located geographically close (\cite{cooper2019price}). Therefore, I define two types of overlap: overlap with an otherwise non-affiliated hospital in the same market, or overlap with an otherwise unaffiliated hospital in a different market. I use the market definitions developed by Dartmouth Atlas, Hospital Referral Regions (HRRs), which are meant to capture geographic regions of the average patient's choice set of hospitals.


    \subsection{Documenting Overlapping Board Members}

    In Figure \ref{fig:connected_percent} I show the prevalence of overlapping board members or executives over time. In 2017, approximately 5\% of hospitals have an overlapping board members or executive with another hospital in the same market that is not formally affiliated otherwise. That is, these hospital are not under common ownership. This percentage increases over time, with 10\% of hospitals sharing a key leader in their market in 2022. I also graph the percent of HRR markets where hospitals within that market share board members or executives. Approximately 15\% of HRRs are potentially affected by this type of affiliation. Sentence here about putting this into perspective relative to hospital consolidation as a whole. 
    
    \begin{figure}[ht!]
        \centering
        \vspace{9mm}
        \caption{Prevalence of Board/Executive Interlock}
        \includegraphics[width=.9\textwidth]{Objects/connected_percent.pdf}
        \label{fig:connected_percent}
    \end{figure}


    Next, I present the geographic distribution of these pairs in Figure \ref{fig:connected_maps}. The blue lines represent hospitals that share a common board member, and the red dots are all other hospitals. Connected hospitals are distributed fairly evenly across the US, apart from relatively few connections on the west coast relative to the population. Over time, the connections remain fairly consistent geographically. 

    \begin{figure}[ht!]
        \centering
        \caption{Geographic Distribution of Board/Executive Interlock}
        \includegraphics[width=\textwidth]{Objects/connected_maps.pdf}
        \label{fig:connected_maps}
    \end{figure}

    Hospitals in the sample are either always connected, never connected, or become connected at some point in time. I show characteristics of hospitals in these categories in Table \ref{tab:hosp_group_stats}. 

    \import{Objects}{hosp_group_stats.tex}

    I now present characteristics of pairs of hospitals that are connected by board members or executives in Table \ref{hospital_pair_types}. In the first column, I show pair characteristics of hospitals connected within the same HRR. In the second column, I present characteristics of hospitals connected to otherwise unaffiliated hospitals located in a different HRR. Therefore, all hospital pairs in the table are connected, but only the first column are connected to hospitals that are potentially direct competitors. Each hospital in a pair is general or specialty (I only consider adult hospitals for these categories), children or adult, and independent or belonging to a system, all characteristics in the AHA data. Additionally, the median bed size of all hospitals in the sample is 101, and I define hospitals as being either small or large depending on whether they fall below or above this median. There are typically 1.5 connections in the same HRR and almost 6 connections in different HRRs. 

    \import{Objects}{hospital_pair_types.tex}

    Most pairs, 76-78\% are two general hospitals. Thus, for the majority of hospitals that have board/executive interlock in the same HRR, the hospital they share a leader with is a direct competitor. I also include characteristics of system versus independent ownership. For there is a nearly even spread of the type of ownership of the pair. 42\% of these pairs are made up of one independent hospital and one system-owned hospital, while 37\% are made up of two independent hospitals and the rest are made up of two hospitals that are both system owned but belong to two different systems. These figures are slightly different for connections in different HRRs, where a larger portion of connections, 45\%, are made up of two system-owned hospitals of two different systems. 
    
    In both same and different HRR connected pairs, 50\% of the connections are such that one hospital is small in bed size and one is large in bed size. The average bed difference is over 200, indicating that the size difference of the average hospital connection is drastic. what's the implication of this? Finally, connected hospitals in the same HRR are approximately 62 kilometers, or 38 miles, away from the other.

    add outcome graphs


    \section{Estimating Anti-competitive Effects of Board Overlap}

    One of the biggest challenges to understanding 


    
    \section{Conclusion}


    This study contributes to the growing literature on hospital governance and market structure by examining the extent and implications of board member overlap across hospitals. By identifying the prevalence of shared board membership, I provide new insights into how governance ties may influence competition, decision-making, and resource allocation in the hospital sector. This paper sheds light on the role of board interlocks in potentially shaping hospital behavior, offering an important perspective on the non-ownership mechanisms that can link hospitals within the same market.

    Using a comprehensive dataset of board members in nonprofit hospitals from 2017-2022, I document the characteristics of hospitals with shared board members and analyze how these overlaps have evolved over time. I identify connections between hospitals and examine the patterns of overlap by hospital type, market, and geographic region. The results indicate that a significant share of hospitals are connected through board interlocks, and a significant portion of hospital markets contain board overlap within the market. Further, I show geographic distributions of overlap and characterize connected hospital pairs by important characteristics such as ownership, size, and specialty. 

    These findings hold important implications for policymakers and future research. Regulators and antitrust authorities should consider the potential competitive effects of board member overlap when assessing hospital market dynamics, particularly in regions with high consolidation or limited provider choice. Future research could further investigate how these governance connections influence hospital pricing, quality of care, and patient outcomes. Expanding this work to explore causal relationships and the broader economic consequences of board interlocks would provide valuable insights for competition in health care.
  

    \newpage


    \printbibliography


    

    

    

    

    

	
	
	


\end{document}