\documentclass[12pt]{article}

\usepackage{tgtermes}
\usepackage{epsf}
\usepackage{epstopdf}
\usepackage{amsmath}
\usepackage{graphicx}
\usepackage{booktabs}
\usepackage[colorlinks=true,linkcolor=blue,citecolor=blue]{hyperref}
\usepackage{dcolumn}
\usepackage{amsmath, amsthm, amssymb}
\usepackage{mwe}
\usepackage{url}
%\usepackage{harvard}
\usepackage{fancyheadings}
\usepackage{longtable}
\usepackage{authblk}
\usepackage{setspace}
%\usepackage[nomarkers]{endfloat}
\usepackage{float}
\usepackage{bbm}
%\usepackage{titling}
\usepackage{subcaption}
\usepackage{algorithm}
\usepackage{algorithmic}
\usepackage{import}
\usepackage[backend=biber,style=authoryear,
sorting=nyt,citestyle=authoryear]{biblatex}
\addbibresource{papercitations.bib}
%\usepackage[nomarkers,nofiglist,notablist]{endfloat}
\usepackage{subcaption}
\usepackage{caption}

\onehalfspacing
\textwidth 6.5in \oddsidemargin 0in \evensidemargin -0.6in
\textheight 8.5in \topmargin -0.2in

\newcolumntype{L}[1]{>{\raggedright\let\newline\\
		\arraybackslash\hspace{0pt}}m{#1}}
\newcolumntype{C}[1]{>{\centering\let\newline\\
		\arraybackslash\hspace{0pt}}m{#1}}
\newcolumntype{R}[1]{>{\raggedleft\let\newline\\
		\arraybackslash\hspace{0pt}}m{#1}}
\newcolumntype{P}[1]{>{\raggedright\tabularxbackslash}p{#1}}

\newtheorem{theorem}{Theorem}[section]
\newtheorem{corollary}[theorem]{Corollary}
\newtheorem{proposition}[theorem]{Proposition}
\newtheorem{lemma}[theorem]{Lemma}

\captionsetup{justification=centering,singlelinecheck=false}


\newcommand{\xsub}[1]{%
	\mbox{\scriptsize\begin{tabular}{@{}c@{}}#1\end{tabular}}%
}

%\renewcommand{\thetable}{\Roman{table}}

\begin{document}
	
	
	
	
	\linespread{1.2}\title{\vspace{-0.5in} Beyond Mergers: Informal Governance Ties in U.S. Hospitals} 
	
	\date{\today}
	
	\author{\vspace{10mm}Hanna Glenn\footnote{School of Economics, University of Queensland, hanna.glenn@uq.edu.au} }
	
	\maketitle
	%\setlength{\droptitle}{-10pt}
	
	\vspace{-0.2in}
	
	\singlespacing\maketitle


 \vspace{3mm}
	
\begin{abstract}
\small{I examine non-merger affiliations among U.S. nonprofit hospitals by constructing a measure of leadership overlap: shared directors or executives across otherwise unaffiliated hospitals. I document that 10\% of hospitals share a board member with a competitor in the same Hospital Referral Region (HRR) and that 25\% of HRRs contain at least one such affiliation. Using stacked difference-in-differences event studies, I first describe associations between overlap and consolidation, finding that losing a leadership tie often coincides with closure or system entry. I then estimate the effect of gaining or losing local overlap on hospital behavior by comparing hospitals that experience an overlap event in the same market to those that experience overlap in a different market, isolating the incremental effect of local connections relative to non-local affiliations. Under standard assumptions, these estimates suggest that local overlap does not affect patient volumes or capital purchases but is followed by increases in service offerings and operating expenses, consistent with strategic expansion or joint ventures. These results highlight the prevalence of informal governance ties and their role in shaping hospital behavior, while associations with consolidation suggest that leadership overlap may serve as an early signal of structural change.}
\end{abstract}
	
	
	
	
	

	
	\onehalfspacing
	
	\newpage

    \section{Introduction}

    Market concentration has increased across industries in the last two decades (\cite{grullon2019us}), sparking renewed debate and, more recently, a shift toward stricter antitrust enforcement. Modern antitrust policy primarily focuses on mergers and acquisitions among large firms. Since the Hart–Scott–Rodino Act of 1976, the U.S. government has reviewed certain transactions that may threaten competition, defined by regulators as “a process of rivalry that incentivizes businesses to offer lower prices, improve wages and working conditions, enhance quality and resiliency, innovate, and expand choice” (\cite{DOJFTC2023MergerGuidelines}). Formal challenges remain rare: since 2015, roughly 3\% of proposed mergers have faced extended review, and 32 proposals were dissolved or restructured (\cite{ConversableEconomist2025Antitrust}).

    Careful scrutiny of large-scale mergers is important. However, growing empirical and policy concern centers on unreported, small-scale mergers and non-merger affiliations. Research shows that small-scale mergers, often called “stealth mergers”, can reduce competition (\cite{majerovitz2021consolidation}; \cite{aggarwal2025stealth}; \cite{wollmann2019stealth}; \cite{wollmann2020get}; \cite{kepler2023stealth}). Non-merger affiliations encompass structural ties such as joint ventures, management consolidation, and overlapping boards of directors, also referred to as board interlock, that do not trigger merger review. Among these, board overlap is particularly noteworthy because it creates direct governance links between firms, is explicitly addressed under the Clayton Act, and has drawn increasing policy attention. While overlapping boards are documented (\cite{LemleyVanLoo2025OverlappingDirectors}; \cite{Poberejsky2025InterlocksInnovation}), their competitive effects remain largely hypothesized.

    This issue is especially relevant in health care, where consolidation is a persistent concern and several states have begun regulating small-scale mergers and non-merger affiliations (\cite{ncsl2024healthcare}). Additionally, in September 2025, the FTC announced the resignation of three directors from Sevita Health after uncovering board overlap with Beacon Specialized Living Services, both providers of residential care for individuals with intellectual and developmental disabilities, citing concerns under the Clayton Act and urging firms to proactively avoid such conflicts (\cite{ftc2025sevita}). Board interlocks are no longer abstract governance issues but active antitrust targets. Thus, in this paper, I construct a measure of non-merger affiliation among U.S. nonprofit hospitals that captures both interlocking boards and executive cross-board service, and I examine how these affiliations are related to consolidation and hospital behavior.

    I collect data on nonprofit hospital executives and board members from publicly available IRS Form 990 filings and merge these records with the American Hospital Association (AHA) Annual Survey. Using this combined dataset, I classify hospital affiliations based on leadership teams and geographic market. I define a non-merger affiliation, or leadership overlap, as the presence of a common board member or an executive serving on another hospital’s board, where the two hospitals are otherwise unaffiliated. That is, they do not belong to the same system or network but share leadership ties. Because the hypothesized effects of non-merger affiliations are anti-competitive, whether the affiliated hospitals compete for the same patients is a critical distinction. To capture this, I separate affiliations into those occurring within the same Hospital Referral Region (HRR) and those across different HRRs, where HRRs represent geographic hospital markets (\cite{Wennberg1973SmallAV}). I show that 10\% of nonprofit hospitals in my sample share a leader with an otherwise unaffiliated hospital in the same market, and 25\% of HRRs contain at least one such affiliation. I consider several hospital outcomes as behaviors that are informed by the corporate governance literature and hospital consolidation literature. I group these outcomes into categories of consolidation, expenses, and product differentiation.

    Similarly to the merger literature, selection into the decision to become affiliated complicates causal inference. Hospitals that gain leadership overlap may differ systematically from those that do not, and these differences could bias estimates if not addressed. To examine consolidation outcomes such as closures and mergers, I use a stacked difference-in-differences (DiD) event-study design to describe associations between overlap and structural change. I find that losing a leadership connected hospital in a different HRR is associated with an increased likelihood of closing, and that losing a leadership connected hospital in the same HRR coincides with becoming part of a system. These estimates should not be interpreted as causal because governance changes may anticipate consolidation or respond to unobserved shocks. Nevertheless, documenting these patterns is important for understanding whether leadership networks and consolidation move together.

    For other hospital behaviors, I adopt the same stacked DiD framework but restrict the sample to hospitals that eventually gain overlap either in the same HRR or in a different HRR. Within this restricted sample, the treatment group consists of hospitals that gain overlap in the same HRR, while the comparison group includes hospitals that gain overlap in a different HRR but have not yet gained same-HRR overlap at the event time. This design improves comparability by conditioning on a shared propensity for connectivity and mitigates bias from permanent differences between hospitals that never connect and those that do. Under standard parallel trends and no-anticipation assumptions, these estimates can be interpreted as the causal effect of gaining local overlap on hospital behavior conditional on gaining any connection. Using this strategy, I show that overlap events are not correlated with aggregate patient populations or hospital purchases, but do pre-empt changes in the number of services offered by the hospital and changes in total operating expenses. These findings are consistent with board overlap representing larger hospital affiliations such as joint ventures or clinical agreements. 


    Prior work establishes that corporations are deeply interconnected through overlapping directors and that these ties shape governance practices and firm performance (\cite{dooley1969interlocking}; \cite{ferris2003too}; \cite{field2013busy}; \cite{bizjak2009option}; \cite{bouwman2011corporate}; \cite{chiu2013board}; \cite{cai2014board}; \cite{baran2017director}; \cite{geng2021does}; \cite{schonlau2009board}; \cite{cai2012board}; \cite{cheng2021social}). That literature largely centers on publicly traded firms and frames overlap as either a governance friction (busy directors, option backdating, strategy convergence) or a source of private benefits (director networks that raise value, profitability, and post‑M\&A performance). In contrast, I study non‑merger affiliations in nonprofit hospitals, an arena often presumed less prone to anti-competitive behavior (\cite{baer2014clayton}; \cite{aai2013section7}) despite mounting evidence of competitive risks in health care (\cite{hulver2023ftc}). I broaden the construct beyond traditional board interlock to include executive cross‑board service, and I move the focus from governance proxies and firm‑level valuation to market‑relevant behaviors that are informed by the literature on both governance and consolidation (\cite{baran2017director}; \cite{geng2021does}; \cite{schonlau2009board}; \cite{cai2012board}; \cite{cheng2021social}). Methodologically, I leverage a stacked differences-in-differences design that distinguishes affiliations formed within versus across geographic markets to isolate the incremental, competition‑salient effect of overlap, rather than inferring behavior from financial outcomes alone. 

    A large empirical literature examines the effects of consolidation in health care, particularly through hospital mergers and system integration. Early work found little evidence that consolidation improved operating costs, efficiency, or quality of care (\cite{alexander1996short}; \cite{ho2000hospital}; \cite{dranove2003hospital}). Yet from 2000 to 2020, hospital mergers surged, leaving a quarter of markets with no independent hospitals by 2020 (\cite{ElevanceHealth2023}). This trend spurred extensive research on price, quality, and efficiency. Studies consistently document modest cost and efficiency gains for hospitals joining larger systems (\cite{schmitt2017hospital}; \cite{andreyeva2024corporatization}; \cite{craig2021mergers}), but these gains rarely translate into lower prices; instead, mergers typically lead to significant price increases as hospitals gain bargaining power with insurers (\cite{gaynor2012impact}; \cite{boozary2019association}; \cite{cooper2019price}; \cite{andreyeva2024corporatization}). Evidence on quality is mixed: most studies find little effect on readmissions or mortality (\cite{haas2011mergers}; \cite{beaulieu2020changes}), though some report increased treatment intensity and mortality for heart disease patients (\cite{hayford2012impact}) or higher readmission rates overall (\cite{andreyeva2024corporatization}). Beyond price and quality, research highlights mechanisms such as convergence in treatment styles (\cite{eliason2020acquisitions}), consolidated service offerings (\cite{mariani2022impact}), and redistribution of patients within markets, including declines in Medicaid admissions in highly concentrated areas (\cite{desai2023hospital}). While this literature focuses on mergers and system affiliation, my study examines a different form of structural connection: non-merger affiliations through board and executive overlap.

    The paper proceeds as follows. In Section \ref{sec:background}, I provide institutional details on nonprofit hospital governance and hospital affiliations. In Section \ref{sec:data}, I outline the data collection process and provide documentation of leadership overlap. In Section \ref{sec:consolidation}, I explore the relationship between leadership overlap and consolidation. Finally, in Section \ref{sec:outcome_results}, I estimate the effect of overlap outcomes on other hospital behaviors. 


    

    


    \section{Background/Setting}\label{sec:background}

    \subsection{Nonprofit Hospital Governance}

    Nonprofits are governed at a high level by a board of directors: a group of unpaid community members that gather to oversee the broad direction and strategies of the firm. The board of directors does not manage the day-to-day activity of the hospital, but provides oversight to the highest level of manager, the Chief Executive Officer, whom the board also selects. The practical governance of boards is highly variable across different hospitals. State-level policies might influence details such as the regularity of board meetings or board member term limits. However, the main function of the board remains consistent: to provide strategic direction and oversight. In research focused on hospital governance, several factors have been found to be correlated with strategic change or hospital-level behaviors, including more influence over the CEO or top management (\cite{golden2001will}; \cite{alexander2008governance}; \cite{jiang2012enhancing}). Additionally, since the Affordable Care Act, boards have been actively involved in quality oversight (\cite{jha2010hospital}; \cite{prybil2014board}; \cite{prybil2010board}). Therefore, while boards are not usually involved in the day-to-day operations of hospitals, they do influence strategy and long-run hospital behavior.

    The board of directors selects a Chief Executive Officer, who then selects other members of the executive team such as a Chief Financial Officer and/or Chief Medical Officer. This leadership team manages the day-to-day operations of the hospital. An executive team usually consists of at least a Chief Executive Officer (CEO) and Chief Financial Officer (CFO), but there is variation in how firms organize these teams. Some hospital executives specialize in health care administration by earning a degree in health care management, or an MBA specific to health care. 

    In this paper, I consider the effects of board and executive interlock, someone sitting as an executive or on the board of multiple organizations. Board overlap does occur in health care with instances of organizations denied merger permission due to boards not acting independently (\cite{huberfeld2006tackling}), of hospitals declaring informal affiliations through leadership (\cite{barnett_babcock_2012}), and of life sciences companies with substantial documented overlap (\cite{manjunath2024illegal}). For many years, oversight of this activity in nonprofits was considered second-order to that of for-profits. However, in 2024, the FTC raised concern over board interlock of two prominent health care organizations (\cite{ftc2025sevita}). Additionally, board will sometimes ask executives from other organizations to sit on their board (\cite{gordon2025ftc}; \cite{alston2024governance}). 

    Early research on board interlock highlights that firms are not isolated entities but interconnected through shared directors (\cite{dooley1969interlocking}), with interlocks linked to firm size, financial ties, and management control. Research has examined governance implications, including whether overlapping boards overextend directors (\cite{ferris2003too}; \cite{field2013busy}), increase stock option backdating (\cite{bizjak2009option}), or promote convergence in governance strategies (\cite{bouwman2011corporate}; \cite{chiu2013board}; \cite{cai2014board}). Empirical evidence focuses largely on financial outcomes: well-connected directors can enhance firm value (\cite{baran2017director}), and board overlap is associated with higher profitability (\cite{geng2021does}), greater acquisition activity, and improved post-merger performance (\cite{schonlau2009board}; \cite{cai2012board}). Mechanisms include shared IT investments (\cite{cheng2021social}), reduced R\&D, and increased product differentiation (\cite{geng2021does}).


    \subsection{Hospital Affiliations}\label{sec:hospaffil}

    Board interlock may be happening in conjunction with affiliation agreements: any relationship between otherwise independent health care providers designed to create shared advantage and value (\cite{aha2018affiliations}). Some examples of affiliation agreements include joint ventures for service provision, clinical agreements among hired health care workers, or joint management. It is common for these types of agreements to involve some sort of shared governance (\cite{aha2018affiliations}; \cite{natlawreview2023ftc}). While some have raised anti-trust concerns over affiliation agreements, they are largely undocumented, and therefore their effects are unknown (\cite{rand2023consolidation}). Organizations do not require pre-authorization to form affiliation agreements, but there are several states that now require notice of any type of affiliation (\cite{ncsl2024healthcare}). 

    The more documented and studied type of hospital affiliation is that of traditional mergers and acquisitions. I will discuss the literature surrounding this topic in order to inform relevant behaviors that could also be affected by affiliation agreements. Hospital consolidation has accelerated over the past two decades, with mergers becoming so prevalent that by 2020, a quarter of hospital markets had no independent hospitals (\cite{ElevanceHealth2023}). Early studies found little evidence that consolidation improved operating costs, efficiency, or quality (\cite{alexander1996short}; \cite{ho2000hospital}; \cite{dranove2003hospital}), but more recent work documents modest efficiency gains for hospitals joining larger systems (\cite{schmitt2017hospital}; \cite{andreyeva2024corporatization}; \cite{craig2021mergers}). These gains, however, rarely translate into lower prices; instead, mergers consistently lead to significant price increases due to enhanced bargaining power with insurers (\cite{gaynor2012impact}; \cite{boozary2019association}; \cite{cooper2019price}; \cite{andreyeva2024corporatization}). Evidence on quality is mixed: most studies find little change in readmission or mortality rates (\cite{haas2011mergers}; \cite{beaulieu2020changes}), though some report higher treatment intensity and mortality for heart disease patients (\cite{hayford2012impact}) and increased readmissions overall (\cite{andreyeva2024corporatization}). Other research highlights mechanisms such as convergence in treatment styles (\cite{eliason2020acquisitions}), changes in service offerings (\cite{mariani2022impact}), and shifts in admission patterns, including reduced Medicaid admissions in concentrated markets (\cite{desai2023hospital}). These findings underscore that consolidation can influence not only costs and prices but also care delivery and patient access.

   
    \section{Data and Summary Statistics}\label{sec:data}

    \subsection{Data on Individuals: Tax Form 990s}\label{sec:990s}

    
   To create a measure of board/executive overlap, I compile data on nonprofit hospital board of directors and executives in years 2013-2023 from publicly available Tax Form 990s. All sufficiently large nonprofits must file these forms with the Internal Revenue Service (IRS) each year. They contain a section in which nonprofits declare their board, executives, and highest compensated employees. This paper focuses only on board members and executives. I limit to firms that fill out Schedule H of the tax form, an indicator that the firm operates a hospital. I extract all names and positions of individuals listed in the section titled ``Officers, Directors, Trustees, Key Employees, and Five Highest Compensated Employees". I also record information on the hospital entity: the Employee Identification Number (EIN), organization name and location, and other characteristics. The initial extraction yields 2,096 nonprofit hospitals.

   Next, I match hospitals in the tax form data to hospitals in other publicly available data sources. The only common information about hospitals across these sources is the name and location. Therefore, I use fuzzy string matching methods to create a crosswalk of EINs to identification numbers in the American Hospital Association (AHA) survey data. Specifically, I compute the Jaro-Winkler distance between hospital names in each data set. Then, I record matches with sufficiently similar names and addresses, yielding a sample of 1,824 matched hospitals. 

    I maintain information about the individuals in the data, including whether they have a title for medical doctor, nurse, or having a degree in health administration. To clean the names, I remove any titles or credentials, and I convert common nicknames to their longer versions for consistency. I also combine names within the same firm that have either slight misspelling differences or, if middle names are only sometimes listed, they have at least 2 names in common. For example, if John Lee Matthew is a board member of hospital A in 2017, and John Matthew is a board member of hospital A in 2018, I assume these names represent the same person. Finally, I use the gender package in R to predict each individual's gender using historical data (\cite{gender}). 

    I present summary statistics of characteristics of the people in the data in Table \ref{tab:boardandexec_people}. There are just under 52,000 people in the data at some point over the course of 10 years, 63\% of which are board members and the remaining 37\% are executives. Board members are slightly more likely to be medical doctors than executives, but the fraction of doctor board members is still relatively low at 18\%. Nurses are consistent across board and executives, but only 3\% of the sample are nurses. A negligible number of people claim to have a health administration degree. Executives are slightly more likely to be female according to the name algorithms, with 38\% of executives estimated to be female and 30\% of board members estimated to be female. I also include average team composition of hospital boards and executive teams, where these trends largely hold at the hospital level. On average, hospital boards contain 19 members and executive teams contain 14 members. 

    \import{Objects}{boardandexec_people.tex}

    I create a list of overlapping board member or executive candidates by identifying any names associated with multiple hospital in the same year. However, it is very common for hospitals within a system to largely share the same board members and/or executives. This is not the type of variation I seek to investigate in this paper, since hospitals within systems are already heavily affiliated. Therefore, I define the relevant type of overlap as someone being on the leadership team of multiple hospitals, where the hospitals are otherwise unaffiliated by system or network. Hospitals within systems are still eligible to have overlap with other hospitals that do not belong to the same system. I exclude hospitals that gain or lose overlapping board members several times in the sample period. 
    
    In the hospital merger literature, mergers have been found to be more significant when the two hospitals involved are located geographically close (\cite{cooper2019price}). Therefore, I define two types of overlap: overlap with an otherwise non-affiliated hospital in the same market, or overlap with an otherwise unaffiliated hospital in a different market. I use the market definitions developed by Dartmouth Atlas, Hospital Referral Regions (HRRs), which are meant to capture geographic regions of the average patient's choice set of hospitals. From these overlap definitions, I characterize hospitals by the first and last year they are connected in the same HRR, and the first and last year they are connected in a different HRR. 


    \subsection{Documenting Leadership Overlap Among Hospitals}

    In Figure \ref{fig:connected_percent}, I illustrate the prevalence of overlapping board members or executives over time. The fraction of hospitals with overlapping leaders rises gradually from 10\% in 2014 to 15\% in 2017, then declines slightly after 2017. The peak in 2017 lacks an obvious explanation, but the subsequent decline may reflect growing consolidation in health care, which reduces opportunities to connect with hospitals outside the same system. I also plot the fraction of markets that include at least two hospitals with overlapping leaders. This measure increases from 15\% in 2014 to nearly 30\% in 2017, then falls to roughly 20\% by 2023. For comparison, about 60\% of hospitals in my sample begin as part of a system, and this fraction rises steadily over time. Although informal affiliations through overlapping leaders occur less often than formal system membership, they remain a notable feature of the U.S. hospital landscape. I also present the geographic distribution of these pairs in Figure \ref{fig:connected_maps}. The blue lines represent hospitals that share a common board member or executive, and the red dots are all other hospitals. Connected hospitals are distributed fairly evenly across the US, apart from relatively few connections on the west coast relative to the population.
    
    \begin{figure}[ht!]
        \centering
        \vspace{9mm}
        \caption{Prevalence of Board/Executive Interlock}
        \includegraphics[width=.9\textwidth]{Objects/connected_percent.pdf}
        \label{fig:connected_percent}
    \end{figure}


    \begin{figure}[ht!]
        \centering
        \caption{Geographic Distribution of Board/Executive Interlock}
        \includegraphics[width=.85\textwidth]{Objects/connected_maps.pdf}
        \label{fig:connected_maps}
    \end{figure}

    

    \subsection{Hospital Characteristics}

    Using the crosswalk between the AHA Survey identifiers and Tax identifiers discussed in Section \ref{sec:990s}, I merge the data on leadership overlap with other hospital characteristics found in the AHA Survey. I gather each hospital's number of beds, whether they are general or specialty, whether they belong to a system, whether they are a teaching hospital, the number of nurses they employ, and their location. Based on overlap, hospitals in the sample fall into three categories: always connected, never connected, or connected at some point in time. Table \ref{tab:hosp_group_stats} reports characteristics for each group. Hospitals that never exhibit leadership overlap, whether within or across markets, differ markedly from connected hospitals, with substantially fewer beds and nurses. In contrast, hospitals that become connected during the sample period resemble connected hospitals much more closely. Over 300 hospitals gain overlap within the same HRR, and almost 400 gain overlap across different HRRs. These groups are not mutually exclusive, so some hospitals experience overlap in both settings.\footnote{I account for this in the estimation.} Nearly 600 hospitals remain connected across different HRRs throughout the sample, making this form of overlap the most common.

    \import{Objects}{hosp_group_stats.tex}

    I now present characteristics of connected pairs of hospitals in Table \ref{hospital_pair_types}. In the first column, I show pair characteristics of hospitals connected within the same HRR. In the second column, I present characteristics of hospitals connected to otherwise unaffiliated hospitals located in a different HRR. Therefore, all hospital pairs in the table are connected, but only the first column are connected to hospitals that are potentially direct competitors. Each hospital in a pair is general or specialty (I only consider adult hospitals for these categories), children or adult, and independent or belonging to a system, all characteristics in the AHA data. Additionally, the median bed size of all hospitals in the sample is 119, and I define hospitals as being either small or large depending on whether they fall below or above this median. There are typically 1.7 connections in the same HRR and almost 8 connections in different HRRs. 

    \import{Objects}{hospital_pair_types.tex}

    I will first discuss hospital pairs located in the same HRR. Most of these pairs, 64\%, consist of two general hospitals, and are thus competitors in terms of typical services provided. An additional 12\% of these pairs are between a general and a specialty hospital, and the remaining 1\% are between two specialty hospitals. Similarly, three quarters of the pairs in the same HRR are between two adult hospitals. While the largest fraction of these pairs, 44\%, is between two independent hospitals, there is also a significant portion, 40\%, that occur between one independent hospital and one system-affiliated hospital. Only 16\% of these pairs are between two hospitals belonging to different systems. Almost half of the pairs are between one small and one large hospitals, and the average bed difference is over 200, indicating a significant size difference on average. Finally, pairs within the same HRR are typically 35 miles apart.

    Now, I discuss hospitals that share leaders with hospitals in different markets. 80\% of these pairs are between two general hospitals, and 90\% are between two adult hospitals. Thus, it is likely that hospitals are seeking expertise from other hospitals similar to theirs moreso than in the same HRR. There is a fairly even spread of whether the hospitals in the pair are independent. Still, 50\% of pairs are between one large and one small hospital, and the bed size difference is 238 on average. As expected, the hospitals in these pairs are located much farther apart geographically than those in the same HRR. 

    \subsection{Hospital Outcomes}\label{sec:outcomes}

    Informed by both the board interlock literature and the hospital consolidation literature, I consider three categories of outcomes that could theoretically be affected by leadership overlap: quality, consolidation and investment decisions, and differentiation. These variables are constructed using the AHA Survey and the Healthcare Cost Report Information System (HCRIS) financial data.

    % The hospital consolidation literature highlights both price and quality of care as key factors in the industry. Mergers often grant hospitals greater market power, enabling them to negotiate higher prices with insurance companies. It is unclear whether shared leaders would lead to similar price increases. If overlap facilitates information sharing about negotiations, it could indirectly influence pricing. However, direct board involvement in pricing decisions seems less likely. As price negotiations are unobservable with the data used in this project, I do not focus on this as a likely outcome. However, quality of care is observable and there are direct actions related to overlap that could lead to changes in quality, such as convergence in treatment styles or investment in technology. There is also an indirect path through which overlapping leaders might impact quality, which is that hospitals might compete for the same pool of experienced board members. If the board members with the most expertise are in high demand, sharing that expertise among multiple hospitals could facilitate greater improvements in quality.  [ADD SENTENCE ABOUT SPECIFIC OUTCOMES I USE]

    In the literature on board interlock, several hospital behaviors are shown to be associated with established firm affiliation: mergers, investments in technology and capital, and product differentiation. Thus, I expand these ideas to the hospital industry. First, I measure hospital consolidation choices using indicators for whether the hospital closes or becomes part of a system. A hospital is considered "closed" if they drop out of the tax data, the AHA Survey, and the HCRIS data at the same time and remain missing for the rest of the sample.\footnote{I drop any hospitals that lose connectedness due to closing.} I use the AHA Survey system identifiers to determine whether a hospital becomes part of a system. Second, I measure changes in investment using several variables from the HCRIS financial data: total operating expenses, capital  purchases (movable equipment, fixed equipment, and building purchases), and health information technology purchases. Finally, previous research has shown that sharing information among competing firms can lead to more specialization. In hospitals, there are two distinct types of differentiation: either through services offered, or the types of patients admitted. Both of these behaviors are key findings in the hospital merger literature, indicating that affiliations through leadership could also facilitate changes in these areas. I measure changes in services offered using questions in the AHA Survey capturing whether specific services are offered in the facility, and how many beds are devoted to each service. I create a variable capturing the total number of services offered. I also define a measure of bed concentration using a standard Herfindahl-Hirschman Index, which captures if a hospital becomes more specialized in the services they offer. This is calculated by summing the squared share of beds in each service $k$:  

        $$\text{bed concentration} = \sum_{k}\left(\frac{\text{beds}_k}{\text{total beds}}\right)^2.$$

    
    Similarly to mergers and acquisitions, hospital decisions to become connected to another hospital by sharing board members or executives are endogenous. Thus, when thinking about how overlap affects the hospital behaviors discussed above, the relevant comparison group is important. Thus, I present trends over time for a select number of outcomes for four different groups of hospitals: (1) those that gain overlap in the same HRR, (2) those that gain overlap in a different HRR, (3) those that are always connected in a different HRR, and (4) those that are never connected. I show these trends for specific key outcomes in Figure \ref{fig:outcome_desc_graph}. 

    \begin{figure}[ht!]
        \centering
        \caption{Select Outcomes by Overlap Status}
        \includegraphics[width=.8\textwidth]{Objects/outcome_desc_graph.pdf}
        \label{fig:outcome_desc_graph}
    \end{figure}

  These trends show that not all hospitals serve as a suitable comparison group for those that gain a connection within the same HRR. In the top panel, which plots the average number of services over time, hospitals that remain unconnected or always connected in a different HRR follow a different trajectory than hospitals that become connected. A similar pattern appears for discharges: hospitals that gain overlap track a comparable path before the connection, then diverge, while hospitals that never gain overlap exhibit a distinct trend of declining Medicare and Medicaid discharges. Finally, total operating expenses reveal entirely different slopes for hospitals that never gain overlap compared to those that do, whether in the same or a different market. For this reason, when estimating the effect of gaining overlap in the same market, I compare these hospitals only to those that become connected in a different market. This approach allows me to isolate the effect of connecting with a competitor from the broader effect of becoming connected.
    


    \section{Overlap and Consolidation}\label{sec:consolidation}

    While hospitals often publicly declare that their leadership affiliations are not intended to lead to mergers or other consolidation (add citations), the underlying nature of overlapping leadership with regards to other types of hospital affiliations is an open empirical question. Theoretically, there are three ways that leadership overlap and consolidation could be related. First, leadership overlap may act as a substitute for formal mergers. In this case, hospitals leverage shared governance to achieve some benefits of consolidation, such as strategic alignment, resource coordination, and cost efficiencies, without incurring the regulatory scrutiny and transaction costs associated with a merger. If this hypothesis holds, we would expect leadership overlap to reduce the likelihood of subsequent merger activity. Second, leadership overlap may serve as a precursor to merging. Hospitals might begin by sharing board members or executives to build trust, harmonize decision-making, and prepare for operational integration. Under this interpretation, overlap is an intermediate step toward full consolidation, and its presence should increase the probability of a future merger. Third, leadership overlap may result from intentions to merge. Here, overlap is not an independent strategic choice but rather a signal of an already-decided merger plan. This scenario introduces endogeneity concerns, as leadership overlap and merger intentions are jointly determined. In addition to its possible role in mergers, leadership overlap may also influence the likelihood of hospital closure. Shared leadership could provide access to broader networks, financial resources, and managerial expertise, which might help struggling hospitals remain viable and avoid closure. Conversely, overlap could signal strategic realignment within a system, where weaker hospitals are deliberately phased out to concentrate resources in stronger facilities. In this sense, leadership overlap could either reduce closure risk by stabilizing operations or increase it if consolidation strategies involve selective shutdowns.

    I begin by examining the relationship between leadership overlap and consolidation outcomes, specifically hospital closures and mergers. For this analysis, I estimate separate event-study models for gaining or losing overlap within the same Hospital Referral Region (HRR) and gaining or losing overlap in a different HRR. Every other hospital in the sample serves as a comparison group, except that I exclude hospitals that mechanically lose overlap because of closure to avoid contaminating the event window. This broader comparison captures how overlap correlates with consolidation across the full market landscape.

    \subsection{Estimation}

    To accommodate staggered treatment timing and allow for dynamic effects, I use a stacked difference-in-differences (DiD) event-study design. For each treated hospital, I construct an event-time panel centered on the first (or last) year of overlap and stack these panels into a single dataset. This approach avoids the bias that arises in traditional two-way fixed effects models when treatment effects are heterogeneous across cohorts or over time, and it provides a flexible way to visualize pre-trends and post-event dynamics. The estimation equation is:

    \begin{equation}\label{eq:consolidation}
    Y_{it} = \alpha + \sum_{k \neq -1} \beta_k \cdot D_{i,t+k} + \gamma_i + \delta_t + \varepsilon_{it},
    \end{equation}

    \noindent where $Y_{it}$ is an indicator for closure or merger for hospital $i$ in year $t$, $D_{i,t+k}$ marks event time $k$ relative to the year of gaining or losing overlap (same HRR or different HRR), and $\gamma_i$ and $\delta_t$ are hospital and year fixed effects. I estimate separate sets of $\beta_k$ for same-HRR and different-HRR overlap events.

    Although I use an event-study framework, these estimates should not be interpreted as causal effects of leadership overlap on consolidation. Reverse causality is a central concern: hospitals may gain or lose overlapping leaders precisely because they anticipate a merger or closure, or because system-level governance changes precede both overlap and consolidation. In addition, unobserved shocks—such as financial distress or strategic realignment—could jointly influence leadership networks and consolidation outcomes. While I control for changes in system ownership and in-system leadership overlap, these adjustments cannot fully eliminate endogeneity. Despite these limitations, understanding whether leadership overlap and consolidation move together is important for characterizing governance dynamics in hospital markets. If overlap consistently precedes consolidation, even without a strict causal interpretation, it may signal that leadership networks serve as an early indicator of structural change. Conversely, if consolidation occurs independently of overlap, this suggests that governance ties and formal integration follow distinct trajectories, or that leadership overlap may serve as a substitute for formal integration.

    \subsection{Results}

    Figure \ref{fig:closure} plots event-study estimates of the relationship between leadership overlap and hospital closure for hospitals that gain or lose overlap, separately for same-HRR and different-HRR events. The vertical axis shows the average treatment effect on the treated (ATT) with 90\% confidence intervals, and the horizontal axis represents event time relative to the year of overlap change.

        \begin{figure}[ht!]
        \centering
        \vspace{9mm}
        \caption{Leadership Overlap and Closure Association}
        \includegraphics[width=.9\textwidth]{Objects/closure.pdf}
        \label{fig:closure}
    \end{figure}

    For hospitals that gain overlap within the same HRR, the estimated effects on closure remain close to zero throughout the event window, with no evidence of systematic pre-trends. This suggests that gaining local overlap is not strongly associated with subsequent closure. In contrast, hospitals that lose overlap in the same HRR exhibit a modest increase in closure risk in later years, although confidence intervals are wide and estimates are imprecise. Patterns differ for hospitals connected across different HRRs. Gaining overlap in a different HRR shows little association with closure, similar to the same-HRR case. However, losing overlap in a different HRR is followed by a sharp increase in closure risk beginning two years after the event, reaching an ATT of almost 0.20 by year four. Compared to the average closure rate of 13\%, this is an extremely large jump in the likelihood of closing. This pattern is consistent with the idea that losing governance ties, particularly those outside the local market, may signal financial distress or strategic withdrawal, though causality cannot be established.

    Similarly, in Figure \ref{fig:independent} I present event-study estimates of the relationship between leadership overlap and the likelihood of being independent (not in a system), separately for hospitals that gain or lose overlap in the same HRR and in different HRRs. 

    \begin{figure}[ht!]
        \centering
        \vspace{9mm}
        \caption{Leadership Overlap and Correlation with Independence}
        \includegraphics[width=.9\textwidth]{Objects/independent.pdf}
        \label{fig:independent}
    \end{figure}

    For hospitals that gain overlap within the same HRR, the estimated effect on independence is small and positive in the years following the event, peaking at roughly 0.05, but confidence intervals include zero throughout. This suggests little systematic change in independent status after gaining local overlap. In contrast, hospitals that lose overlap in the same HRR exhibit a downward shift in being independent, particularly in the first year after losing overlap, indicating that loss of local governance ties may coincide with integration into a system or merger activity. The magnitude is approximately $-0.03$, compared to an average likelihood of being independent of around 40\%. Patterns for different-HRR overlap do not show any meaningful relationship with system affiliation. These results align with the idea that losing in-market leadership connections often accompanies formal consolidation, though causality cannot be established. Reverse causality remains plausible: hospitals may restructure governance in anticipation of joining a system, or system-level changes may drive both overlap loss and independence status.

    Overall, these findings underscore that leadership networks and formal ownership structures evolve together. While the estimates do not imply a direct causal effect, they highlight that changes in overlap, particularly losses, are correlated with consolidation.



    \section{Leadership Overlap and Hospital Outcomes}\label{sec:outcome_results}

    \subsection{Estimation and Identification}

    I estimate the effect of gaining leadership overlap within the same Hospital Referral Region (HRR) on the hospital outcomes discussed in Section \ref{sec:outcomes} using a stacked difference-in-differences event-study design. This approach is well-suited for staggered adoption because it avoids the bias that arises in traditional two-way fixed effects models when treatment effects are heterogeneous across treated cohorts or over time. For each treated hospital, I construct an event-time panel centered on the first year of same-HRR overlap and stack these panels into a single dataset, enabling estimation of dynamic effects.

    To improve comparability, I restrict the sample to hospitals that eventually gain overlap either in the same HRR or in a different HRR. Within this restricted sample, the treatment group consists of hospitals that gain overlap in the same HRR, while the comparison group includes hospitals that do not experience same-HRR overlap during the event window. Conditioning on eventual connectivity in some capacity mitigates bias from permanent differences between hospitals that never connect and those that do, while allowing for meaningful contrasts between hospitals that are competing for the same patients and those that are not. I estimate the following specification on the stacked data:

    \begin{equation}\label{eq:stacked}
    Y_{it} = \alpha + \sum_{k \neq -1} \beta_k \cdot D_{i,t+k} + \gamma_i + \delta_t + \varepsilon_{it},
    \end{equation}

    \noindent where $Y_{it}$ is the outcome for hospital $i$ in year $t$, $D_{i,t+k}$ is an indicator for event time $k$ relative to the year of gaining same-HRR overlap (with $k = -1$ as the omitted category), $\gamma_i$ and $\delta_t$ are hospital and year fixed effects, respectively. The coefficients $\beta_k$ trace dynamic effects before and after the event.

    To interpret $\beta_k$ as the causal effect of gaining same-HRR overlap, I rely on several assumptions. First, hospitals that gain overlap in the same HRR would have followed similar outcome trends as hospitals that gain overlap in a different HRR in the absence of treatment, satisfying a parallel trends condition within the restricted sample. Evidence from Figure \ref{fig:outcome_desc_graph} and the absence of significant pre-event coefficients supports this assumption. Second, I assume there is no anticipation of treatment. I discuss this assumption in conjunction with the results. Third, hospitals’ outcomes should depend only on their own overlap status, not on changes at other hospitals in the same HRR or in the comparison group. This no-spillover assumption is strong in settings where leadership networks extend across markets, and violations could bias estimates if overlap at one hospital affects competitors indirectly. Finally, I assume that no other events correlated with both leadership overlap and the relevant outcomes occur; I address this by controlling for changes in system ownership and in-system leadership overlap.

    Under these assumptions, the estimated event-time coefficients capture the dynamic effect of gaining same-HRR overlap relative to hospitals that gain overlap in a different HRR but have not yet gained same-HRR overlap. This effect is conditional on eventual connectivity and should be interpreted as the impact of local overlap among hospitals that share similar connectivity trajectories.


    %\subsection{Overlap and Quality}

    \subsection{Overlap and Expenses}

    Figure \ref{fig:expenses} presents event-study estimates for two expense measures: purchases of buildings or equipment and total operating expenses. The vertical axis shows the average treatment effect on the treated (ATT) with 90\% confidence intervals, and the horizontal axis represents event time relative to the year of gaining or losing overlap.

    For purchases of buildings or equipment, hospitals that gain or lose overlap exhibit no clear pattern, with estimates fluctuating around zero and confidence intervals that include zero throughout the event window. Total operating expenses display a different dynamic. Hospitals that lose overlap show a pronounced rise in operating expenses around the event, followed by a steep decline by year four. This behavior is consistent with leadership overlap representing joint ventures or management consolidation, and if overlap ends then the hospital takes on more of the operational expenses. 


    \begin{figure}[h!]
    \centering
    \caption{Expense Outcomes\label{fig:expenses}}
    \begin{subfigure}[b]{0.47\textwidth}
        \caption{Purchases (Building or Equipment)}
        \includegraphics[width=\textwidth]{Objects/any_purch.pdf}
        \label{fig:any_purch}
    \end{subfigure}
    \begin{subfigure}[b]{0.47\textwidth}
        \caption{Total Operating Expenses}
        \includegraphics[width=\textwidth]{Objects/tot_operating_exp.pdf}
        \label{fig:tot_oper_exp}
    \end{subfigure}
    \end{figure}

    \subsection{Overlap and Differentiation}

    Figure \ref{fig:mcaid_mcare} plots event-study estimates of the relationship between leadership overlap and hospital discharges, including Medicaid, Medicare, and total discharges. The vertical axis shows the average treatment effect on the treated (ATT) with 90\% confidence intervals, and the horizontal axis represents event time relative to the year of gaining or losing overlap. Across all three panels, the estimated effects of gaining or losing overlap on patient volumes are small and statistically indistinguishable from zero. For Medicaid discharges, point estimates fluctuate slightly after the event, but confidence intervals are wide and include zero throughout the window. Medicare discharges show a similar pattern, with no systematic increase or decrease following changes in overlap. Total discharges exhibit more variability, particularly for hospitals that gain overlap, but the estimates remain imprecise and do not indicate a consistent trend.

    \begin{figure}[h!]
    \centering
    \caption{Patient Population Outcomes\label{fig:mcaid_mcare}}
    \begin{subfigure}[b]{0.47\textwidth}
        \caption{Medicaid Discharges}
        \includegraphics[width=\textwidth]{Objects/mcaid_discharges.pdf}
        \label{fig:mcaid}
    \end{subfigure}
    \begin{subfigure}[b]{0.47\textwidth}
        \caption{Medicare Discharges}
        \includegraphics[width=\textwidth]{Objects/mcare_discharges.pdf}
        \label{fig:mcare}
    \end{subfigure}
    \begin{subfigure}[b]{0.47\textwidth}
        \caption{Total Discharges}
        \includegraphics[width=\textwidth]{Objects/tot_discharges.pdf}
    \end{subfigure}
    \end{figure}

    Figure \ref{fig:services} shows event-study estimates for two additional dimensions of product differentiation: bed concentration and the number of distinct services offered. Bed concentration exhibits no systematic change following overlap events. Estimates remain close to zero for both gaining and losing overlap, and confidence intervals include zero throughout the event window. This suggests that leadership overlap does not meaningfully alter the distribution of beds across service lines. In contrast, the number of services displays a clear pattern. Hospitals that gain overlap expand their service offerings in the years after the event, with point estimates rising steadily to approximately 35 additional services 2-4 years after gaining overlap. Conversely, hospitals that lose overlap show a decline in service breadth, decreasing the number of services by 30 in the first year after gaining a connection in the same HRR. Relative to 75 services offered on average, this is a meaningful change in products offered by the hospital. These results are consistent with the idea that governance ties influence strategic positioning: gaining overlap may facilitate coordination or resource sharing that enables hospitals to broaden their portfolio, while losing overlap may constrain capabilities or signal the end of a joint venture.
     
    \begin{figure}[h!]
    \centering
    \caption{Service Offering Outcomes\label{fig:services}}
    \begin{subfigure}[b]{0.47\textwidth}
        \caption{Bed Concentration}
        \includegraphics[width=\textwidth]{Objects/bed_conc.pdf}
        \label{fig:bed_conc}
    \end{subfigure}
    \begin{subfigure}[b]{0.47\textwidth}
        \caption{Number of Services}
        \includegraphics[width=\textwidth]{Objects/num_services.pdf}
        \label{fig:num_services}
    \end{subfigure}
    \end{figure}




\section{Conclusion}

This study contributes to the growing literature on hospital governance and market structure by examining the prevalence and implications of leadership overlap across hospitals. By documenting how these governance ties form and evolve, I provide new evidence on an important non-ownership mechanism that links hospitals within and across markets. Leadership overlap may influence strategic decisions, resource allocation, and competitive positioning, yet its role in shaping hospital behavior remains understudied.

Using a comprehensive dataset of nonprofit hospital boards from 2014–2023, I first describe patterns of overlap and their geographic distribution. A substantial share of hospitals are connected through shared leaders, and many markets contain at least two hospitals with overlapping governance. These descriptive findings underscore the potential for informal affiliations to complement or substitute for formal system membership.

I then examine two sets of outcomes. For consolidation, closures and mergers, I estimate event-study models using a stacked difference-in-differences design and interpret the results as associations rather than causal effects. Reverse causality and omitted shocks remain plausible, but the patterns reveal that changes in leadership overlap often coincide with structural transitions, suggesting that governance ties may serve as early indicators of consolidation.

For hospital behaviors, including patient mix, service offerings, and expenses, I use the same event-study framework but restrict the sample to hospitals that eventually gain overlap either in the same HRR or in a different HRR. This design improves comparability and supports a causal interpretation under standard assumptions. The results show that gaining local overlap is associated with expanded service offerings and higher operating expenses, while patient volumes remain largely unchanged. These findings suggest that governance networks influence strategic positioning more than patient flows.

These findings have direct implications for antitrust enforcement. The Federal Trade Commission (FTC) has recently scrutinized shared leadership arrangements, viewing them as potential threats to competition. My results suggest that while leadership overlap often coincides with consolidation, the most significant structural changes, such as mergers and closures, are driven by broader organizational strategies rather than overlap alone. This distinction matters for policy: monitoring governance ties may help identify early signals of consolidation, but enforcement efforts should continue to prioritize large mergers and system expansion, where competitive harm is most likely. Leadership overlap appears to complement, rather than replace, formal integration, reinforcing the need for antitrust authorities to focus on transactions that materially alter market structure.
    \newpage


    \printbibliography


    



    

    

	
	
	


\end{document}