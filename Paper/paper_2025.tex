\documentclass[12pt]{article}

\usepackage{tgtermes}
\usepackage{epsf}
\usepackage{epstopdf}
\usepackage{amsmath}
\usepackage{graphicx}
\usepackage{booktabs}
\usepackage[colorlinks=true,linkcolor=blue,citecolor=blue]{hyperref}
\usepackage{dcolumn}
\usepackage{amsmath, amsthm, amssymb}
\usepackage{mwe}
\usepackage{url}
%\usepackage{harvard}
\usepackage{fancyheadings}
\usepackage{longtable}
\usepackage{authblk}
\usepackage{setspace}
%\usepackage[nomarkers]{endfloat}
\usepackage{float}
\usepackage{bbm}
%\usepackage{titling}
\usepackage{subcaption}
\usepackage{algorithm}
\usepackage{algorithmic}
\usepackage{import}
\usepackage[backend=biber,style=authoryear,
sorting=nyt,citestyle=authoryear]{biblatex}
\addbibresource{papercitations.bib}
%\usepackage[nomarkers,nofiglist,notablist]{endfloat}
\usepackage{subcaption}
\usepackage{caption}

\onehalfspacing
\textwidth 6.5in \oddsidemargin 0in \evensidemargin -0.6in
\textheight 8.5in \topmargin -0.2in

\newcolumntype{L}[1]{>{\raggedright\let\newline\\
		\arraybackslash\hspace{0pt}}m{#1}}
\newcolumntype{C}[1]{>{\centering\let\newline\\
		\arraybackslash\hspace{0pt}}m{#1}}
\newcolumntype{R}[1]{>{\raggedleft\let\newline\\
		\arraybackslash\hspace{0pt}}m{#1}}
\newcolumntype{P}[1]{>{\raggedright\tabularxbackslash}p{#1}}

\newtheorem{theorem}{Theorem}[section]
\newtheorem{corollary}[theorem]{Corollary}
\newtheorem{proposition}[theorem]{Proposition}
\newtheorem{lemma}[theorem]{Lemma}

\captionsetup{justification=centering,singlelinecheck=false}


\newcommand{\xsub}[1]{%
	\mbox{\scriptsize\begin{tabular}{@{}c@{}}#1\end{tabular}}%
}

%\renewcommand{\thetable}{\Roman{table}}

\begin{document}
	
	
	
	
	\linespread{1.2}\title{\vspace{-0.5in} Overlapping Board Members in the Hospital Industry} 
	
	\date{\today}
	
	\author{\vspace{10mm}Hanna Glenn\footnote{Department of Economics, Emory University, 1602 Fishburne Drive, Atlanta, GA 30322, hanna.glenn@emory.edu.} }
	
	\maketitle
	%\setlength{\droptitle}{-10pt}
	
	\vspace{-0.2in}
	
	\singlespacing\maketitle


 \vspace{3mm}
	
    \begin{abstract}
		{\small
    Consolidation in health care has been a heavily discussed topic in research and regulation, as more concentration is linked to higher prices and lower quality. Aside from formal mergers and acquisitions, there is opportunity for hospitals to form affiliations with other competing hospitals through sharing common board members, which has mainly been studied in the context of large public firms. This paper establishes the existence of overlapping board members in competing nonprofit hospitals in the US. Using a data set constructed from IRS Tax Form 990s, I show that 8\% of this sample of nonprofit hospitals have overlapping board members with an otherwise unaffiliated hospital, and 17\% of hospital referral regions have overlapping boards within the market. Having an overlapping board member is correlated with quality metrics and investment decisions, though I do not draw a causal link. These findings highlight an understudied mechanism through which hospitals form affiliations, potentially affecting patient and market outcomes. 
		} 
	\end{abstract}
	
	
	
	
	

	
	\onehalfspacing
	
	\newpage

    \section{Introduction}

    % Consolidation in health care remains a forefront concern for regulators in the US, as both horizontal and vertical integration have substantially increased over the last several decades (\cite{levinson2024ten}). Researchers have documented higher prices (\cite{brot2024pays}), and mixed effects on quality (\cite{kessler2000hospital}; \cite{cooper2011does}) as a result of health care becoming more concentrated through mergers and acquisitions. There is significant antitrust scrutiny in this sector, and proposed mergers are regularly denied due to expected harmful effects to competition (\cite{Meyer2022}). Firms in other industries have been shown to engage in methods of informal affiliation besides formal mergers and acquisitions. For example, instances of board overlap in public firms have been cited to violate the Clayton Act due to their anti-competitive nature. Yet, despite concerns of low competition, this type of connectedness among firms has not been given much consideration in health care. In this paper, I document the extent of overlapping board members in nonprofit hospitals in the US, and explore avenues of future research to investigate 
    % potential anti-competitive effects of this type of affiliation.

    % The presence of overlapping board members across firms, also referred to as board interlock, has been heavily studied in corporate finance, particularly for large publicly traded firms. Board overlap can facilitate indirect collusion on pricing and output strategies, and thus by the Clayton Act of 1914, board overlap among sufficiently large competing firms is illegal. Yet, it is known to exist even among the largest publicly traded firms. Researchers have found that board overlap is correlated with governance practices, performance, investment in research and development, and product differentiation (\cite{cai2014board}; \cite{lamb2016ties}; \cite{geng2021does}). However, it is still unclear whether board overlap really is anti-competitive, as it could also facilitate valuable knowledge sharing and allow multiple firms to gain from expertise of high quality board members. While the Department of Justice has increased scrutiny of board overlap, their focus has remained on tech and private firms (\cite{Morse2023}). Board overlap does occur in health care with instances of organizations denied merger permission due to boards not acting independently (\cite{huberfeld2006tackling}), of hospitals declaring informal affiliations through leadership (\cite{barnett_babcock_2012}), and of life sciences companies with substantial documented overlap (\cite{manjunath2024illegal}). However, little is known about how prevalent board overlap is in the hospital industry, and whether it is correlated with other hospital characteristics. 

    % In this paper, I form networks of hospital board of director members using Tax Form 990s, which are yearly declarations of the financial state of nonprofits in the US that also include information about who is leading the firm. Extracting these identities and matching to other sources of hospital information, I show that approximately 8\% of the nonprofit hospitals in the sample are affiliated with another hospital in the same market through shared board members, even excluding system or network affiliations. Among hospitals that share board members, two thirds of the hospital pairs are between two general adult hospitals. That is, a majority of the shared board members serve on directly competing hospitals in the same geographic market. In addition, half of the hospital pairs are among one large and one small hospital, and half of the hospital pairs are between two hospitals that do not belong to a system. 

    % Informed by established research on board overlap and hospital consolidation, I describe categories of hospital behaviors that may be affected by board overlap: price/quality, investment decisions, and differentiation. I construct various measures of behavior in these categories, and show how they are correlated with board overlap for small/large and system/independent hospitals. While the purpose of this paper is not to establish a causal relationship between board overlap and hospital behavior, there are key observable differences in the average quality and investment decisions of hospitals with and without board overlap that motivate future research on the topic. 

    %  This paper contributes to our understanding of board overlap in a setting not previously considered and our understanding of hospital affiliations apart from formal mergers and acquisitions. The findings have implications for policymakers in assessing potential antitrust violations in certain nonprofit settings, and for future research in identifying the causal effects of board overlap in hospitals.


    \section{Related Literature}\label{sec:relatedlit}


    \subsection{Overlapping Board Members}\label{sec:boardoverlaplit}

    One of the first studies of board overlap argues that firms are not independent, self-sufficient entities as typically assumed in economics, but are actually heavily interconnected via overlapping board members (\cite{dooley1969interlocking}). Dooley frames board interlock as ``the root of many evils", and finds several characteristics to be correlated with the likelihood of sharing a board member with another corporation: size, financial relationships with other companies, and management control. This led to a number of studies focusing on how board interlock affects corporate governance, including whether board overlap stretches individual board members too thin (\cite{ferris2003too}; \cite{field2013busy}), contributes to increases in backdating stock options (\cite{bizjak2009option}), or led to convergence in governance strategy and behavior (\cite{bouwman2011corporate}; \cite{chiu2013board}; \cite{cai2014board}). 

    The literature thus establishes a strong link between board interlock and governance practices. However, the relationship between board overlap and firm behavior, particularly its impact on competition, remains less understood. Theoretically, the direction of this effect is ambiguous. On one hand, board interlock is often viewed as an anti-competitive practice that enables implicit collusion. Alternatively, firms may seek highly experienced and connected individuals for their boards, leading to valuable knowledge sharing and enhanced decision-making. Most studies in this area have focused on the financial performance implications of board overlap, with firm behavior considered primarily as a mediating mechanism.

    Two main studies analyze of the effect of board overlap on financial performance. First, \citeauthor{baran2017director} (\citeyear{baran2017director}) leverages entry and exit from the S\&P 500, along with variation in the local supply of directors, and finds that well-connected board members contribute positively to firm value (\cite{baran2017director}). Similarly, leveraging policies in the US that remove barriers to board overlap, \citeauthor{geng2021does} (\citeyear{geng2021does}) finds that board overlap leads to higher profitability (\cite{geng2021does}). \citeauthor{schonlau2009board} (\citeyear{schonlau2009board}) and \citeauthor{cai2012board} (\citeyear{cai2012board}) analyze financial performance after mergers and acquisitions of publicly traded firms with connected boards, finding that firms with board overlap are more likely to be involved in acquisitions, and that specific types of board connections lead to better financial performance for the acquirer after the transaction (\cite{schonlau2009board}; \cite{cai2012board}). 
    
    Beyond financial performance, several studies explore the relationship between board overlap and firm investment decisions as a mechanism for changes in financial performance. One paper examines whether interconnected firms are more likely to invest in the same information technology, and finds a positive association between board overlap and investment in IT among firms with active boards, indicating that timely communication is an important factor in whether board overlap matters (\cite{cheng2021social}). Finally, \citeauthor{geng2021does} (\citeyear{geng2021does}) finds that lower research and development expenditure and more product differentiation are key mechanisms through which overlapping firms improve profitability (\cite{geng2021does}). 

    Despite these insights, significant gaps remain in our understanding of board overlap. Research has largely focused on governance or firm-level performance, while firm behavior and the broader market implications of board overlap remain underexplored. Additionally, most studies have been limited to large, publicly traded firms. This focus stems from the assumption that nonprofits engage in less anti-competitive behavior (\cite{baer2014clayton}; \cite{aai2013section7}), making board overlap in such organizations less concerning. However, this assumption is flawed, particularly in industries like health care, where nonprofit hospitals dominate and have been shown to engage in anti-competitive practices (\cite{hulver2023ftc}). Further research is needed to assess the competitive effects of board overlap across different market settings, particularly in nonprofit sectors where such dynamics may have substantial implications for industry competition and consumer welfare. I draw from this literature to inform potential outcomes of interest when considering board overlap in hospitals: financial performance, mergers and acquisitions, investment, and product differentiation. I discuss in Section \ref{sec:hospbehaviors} how these outcomes relate to hospitals specifically. 





    \subsection{Common Ownership}\label{sec:commonownerlit}

    As consolidation in investment companies and in many industries has increased over time, so has the prevalence of common ownership among publicly traded firms. That is, firms within the same industry and market are increasingly owned by the same investment companies. While a separate occurrence and literature than that of board overlap, there are numerous parallels between the two concepts. Unlike publicly traded firms, nonprofits do not have shareholders that receive dividend payments. However, in some ways, a board functions similarly to shareholders: they influence the strategic direction of the organization through information sharing and voting on firm decisions without actively managing day-to-day operations. Further, the literature on common ownership in relation to market concentration and economic incentives is more developed than that of board overlap. Thus, I draw on this literature to inform outcomes that are potentially related to board overlap in the hospital industry. 
    
    Whether common ownership affects competition is a widely debated theme in the economics literature. Theoretical models predict that common ownership will lead to lower equilibrium output and higher markups through firms placing non-zero weight on competitors' profit (\cite{rubinstein1983competitive}; \cite{rotemberg1984financial}; \cite{azar2012new}). Empirical evidence has grown tremendously since the seminal studies by \citeauthor{he2017product} (\citeyear{he2017product}) and \citeauthor{azar2018anticompetitive} (\citeyear{azar2018anticompetitive}), who found that common ownership leads to greater consolidation (acquisitions, joint ventures, strategic alliances), and that common ownership led to higher prices in the airline industry, respectively (\cite{he2017product}; \cite{azar2018anticompetitive}). However, these findings have been critiqued for how they measure common ownership and endogeneity concerns (\cite{kennedy2017competitive}; \cite{lewellen2021does}).

    A number of studies have since attempted to estimate the effect of common ownership on a range of outcomes including prices, research and development, merger activity, and patent durability (\cite{gerardi2023critical}). Many of which document a statistically significant relationship between common ownership and the relevant outcome. However, recent studies using alternative identification strategies find that common ownership can actually increase competition through lower insider trading profits, more product development, and higher investments (\cite{chen2023does}; \cite{kini2024common}). Additionally, recent literature has considered common ownership in the context of private firms with venture capitalists, where consolidated ownership is even more prevalent, and has found that competition decreases and information sharing increases in this setting (\cite{lindsey2008blurring}; \cite{gonzalez2020exchanges}; \cite{li2023common}; \cite{eldar2024common}).

    The literature on common ownership yields several conclusions for a study on board overlap. The first is a theoretical foundation confirming the importance of particular outcomes relevant to competition and consumer welfare, including merger and acquisition activity and investment decisions. Second, the common ownership literature focuses much more on how prices are impacted, which is also relevant in the health care setting. Finally, this literature introduces important concerns about the endogenous choice of owners or board members. While I do not seek to causally estimate the effects of board overlap in hospitals in this paper, this is an important consideration for future research. 
    

    \subsection{Consolidation in Hospitals}\label{sec:hospconsol}

    There is a rich empirical literature examining the effects of consolidation in health care that includes both vertical and horizontal integration between hospitals, practices, and physicians. As the focus of this paper is affiliation among competing hospitals, I draw on the literature examining hospital consolidation in particular through mergers and systems. This will inform hospital behaviors that might be correlated with board overlap but are unique to the hospital setting and thus weren't discussed in Section \ref{sec:boardoverlaplit}. The main outcomes of interest in this area of research have centered around price and quality of care. 

    Early studies on hospital consolidation found very little evidence of impacts on operating costs, efficiency, or quality of care (\cite{alexander1996short}; \cite{ho2000hospital}; \cite{dranove2003hospital}). From 2000-2020, the occurrence of hospital mergers increased dramatically, where a quarter of hospital markets included zero independent hospitals by 2020 (\cite{ElevanceHealth2023}). This increase in concentration has led to a vast number of studies estimating its effect on quality, operating costs, and prices. 

    Studies typically document cost and efficiency gains experienced by hospitals who merge to become part of a larger system (\cite{schmitt2017hospital}; \cite{andreyeva2024corporatization}), even if the effects are small (\cite{craig2021mergers}). Naturally, this leads to the question of whether such efficiency gains are passed through to patients by lower prices. Across various settings, data sources, and methodology, hospital mergers are associated with significant increases in price due to hospitals having greater negotiating power with insurance companies (\cite{gaynor2012impact}; \cite{boozary2019association}; \cite{cooper2019price}; \cite{andreyeva2024corporatization}). 
    
    There is mixed evidence on whether such gains in efficiency translate to better quality of care.  Several studies comparing hospitals that became a part of a system with observably similar hospitals that did not merge find little to no effect on readmission or mortality rates on average (\cite{haas2011mergers}; \cite{beaulieu2020changes}). However, one study finds that among heart disease patients, treatment intensity and mortality rates both increase after hospital consolidation (\cite{hayford2012impact}), and another finds increased readmission rates across all types of patients (\cite{andreyeva2024corporatization}). 

    A select number of studies investigate other dimensions that could be affected by more system affiliation over time. In a slightly different setting of acquisitions of dialysis centers, \citeauthor{eliason2020acquisitions} (\citeyear{eliason2020acquisitions}) finds that decreases in quality are driven primarily by the convergence of treatment styles among acquired centers and their acquirer (\cite{eliason2020acquisitions}). Additional studies have suggested that consolidated service offerings is a potential mechanism for effects on quality (\cite{mariani2022impact}). Finally, \citeauthor{desai2023hospital} (\citeyear{desai2023hospital}) examine admission patterns of low-income patients as a result of increased hospital concentration, and find that Medicaid admissions decrease as a result of increased concentration. That is, highly concentrated markets exhibit redistribution patterns of patients within the market (\cite{desai2023hospital}). 

    This literature informs ways that hospital affiliations might affect behaviors and outcomes specific to hospitals. In addition to the behaviors discussed in Sections \ref{sec:boardoverlaplit} and \ref{sec:commonownerlit}, measures of quality of care, treatment styles, service offerings, and admission patterns could also be important aspects driven by affiliation with other hospitals. I discuss these in more detail in Section \ref{sec:hospbehaviors}. 

    

    \subsection{Informing Hospital Behaviors}\label{sec:hospbehaviors}

    The main purpose of this paper is to document the prevalence of board overlap in US hospitals in recent years. While I do not establish any causal relationship between board overlap and hospital behaviors, I consider the literature on board overlap, common ownership, and hospital consolidation discussed in Sections \ref{sec:boardoverlaplit}-\ref{sec:hospconsol} to inform a set of hospital behaviors that may be relevant for future research. I divide these behaviors into three categories: price/quality, investment decisions, and differentiation. 

    The hospital consolidation literature highlights price and quality of care as key factors in the industry. Mergers often grant hospitals greater market power, enabling them to negotiate higher prices with insurance companies. It is unclear whether shared board members would lead to similar price increases. If board overlap facilitates information sharing about negotiations, it could indirectly influence pricing. However, direct board involvement in pricing decisions seems less likely. Nevertheless, with access to data on actual prices charged by hospitals, this remains an interesting outcome for future research to explore.

    It may be more likely that quality of care could be impacted by overlapping board members among hospitals. There are direct actions that could lead to changes in quality such as convergence in treatment styles or investment in technology, which I will discuss in detail in the following sections. There is also an indirect path through which overlapping board members might impact quality, which is that hospitals might compete for the same pool of experienced board members. If the board members with the most expertise are in high demand, sharing that expertise among multiple hospitals could facilitate greater improvements in quality.  

    In both board overlap and common ownership, several investment decisions are shown to be associated with established firm affiliation. First, hospitals with board shared board members might be more likely to merge in the future. Second, sharing board members could facilitate engagement in the same types of technology or capital. Therefore, an area of future research is to look at outcomes such as health information technology investments, specific medical technology advancements in the hospital, and capital investments. 

    Finally, previous research has shown that product differentiation is associated with board overlap in other industries. That is, sharing information among competing firms can lead to more specialization. In hospitals, there are two distinct types of differentiation: either through services offered, or the types of patients admitted. Both of these behaviors are key findings in the hospital merger literature, indicating that affiliations through board members could also facilitate changes in these areas. In future research, one could think about measuring the likelihood of specialized services being offered, where hospitals might shift investments towards services not offered by their affiliated hospital. Additionally, patient composition in the hospital, measured by the payer mix of admitted patients or complexity of their conditions, could be compared among hospitals with and without shared board members. 

    \section{Background/Setting}

    

    \section{Data Construction}

    
   I compile data on nonprofit hospital board of directors in years 2017-2022 from publicly available Tax Form 990s, which all sufficiently large nonprofits must file with the Internal Revenue Service (IRS) each year. These forms contain a section in which nonprofits declare their board, executives, and highest compensated employees. This paper focuses only on the board of director identities. I limit to firms that fill out Schedule H of the tax form, an indicator that the firm operates a hospital. I extract all names and positions of individuals listed in the section titled ``Officers, Directors, Trustees, Key Employees, and Five Highest Compensated Employees". I also record information on the hospital that filed the tax form: the Employee Identification Number (EIN), organization name and location, and other characteristics. The initial extraction yields 2,096 nonprofit hospitals.

   Next, I match hospitals in the tax form data to hospitals in other publicly available data sources. The only common information about hospitals across these sources is the name and location. Therefore, I use fuzzy string matching methods to create a crosswalk of EINs to identification numbers in the American Hospital Association (AHA) survey data. Specifically, I compute the Jaro-Winkler distance between hospital names in each data set. Then, I record matches with sufficiently similar names and addresses, yielding a sample of 1,836 matched hospitals. 

    I limit the individuals in the sample to those on the board of directors, excluding executives that also serve on the board. I then record self-reported information about the board members in the data, including whether they include a title for medical doctor, nurse, or having a degree in health administration. I also record whether the firm records that they are a non-voting member of the board (though this is very rare). Several steps are necessary to ensure the names are clean and consistent across firms to identify people across hospitals. First, I remove any titles or credentials, and I convert common nicknames to their longer versions for consistency. Second, I combine names within the same firm that have either slight misspelling differences or, if middle names are only sometimes listed, they have at least 2 names in common. For example, if John Lee Matthew is a board member of hospital A in 2017, and John Matthew is a board member of hospital A in 2018, I assume these names represent the same person. I use the gender package in R to predict each individual's gender using historical data (\cite{gender}). 

    I create a list of overlapping board member candidates by identifying any names that occur on multiple hospital boards in the same year. However, it is very common for hospitals within a system to largely share the same board. This is not the type of variation I seek to investigate in this paper, since hospitals within systems are already heavily affiliated. Therefore, I define the relevant type of overlap as someone being on the board of multiple hospitals that are otherwise \textit{unaffiliated}. Hospitals within systems are still eligible to have overlap with other hospitals that do not belong to the same system as they do. Further, In the hospital merger literature, mergers have been found to be more significant when the two hospitals involved are located geographically close \cite{cooper2019price}. Therefore, I define two types of overlap: overlap with an otherwise non-affiliated hospital in the same market, or overlap with an otherwise unaffiliated hospital in a different market. All other hospitals have no board overlap with otherwise unaffiliated hospitals. I use the market definitions developed by Dartmouth Atlas, Hospital Referral Regions (HRRs), which are meant to capture geographic regions of the average patient's choice set of hospitals.

    The final sample limitation I make for simplicity is to remove hospitals that gain and then lose overlapping board members in the sample. Since treatment will be turning on overlapping board member, I remove those who "lose" their treatment. 

    \subsection{Documenting Overlapping Board Members}

    The full sample of hospitals 



    \subsection{Summary Statistics}

    First, I present a summary statistics table showing characteristics of all hospitals in the final sample, which consists of 1,467 nonprofit hospitals, almost all of which are adult general hospitals. They have, on average, 178 beds and 14 board members. Almost 40\% of the hospitals in the sample are academic medical centers, almost 70\% of the sample are located in metropolitan areas, and 56\% belong to a system. 

    %\import{Objects}{all_summarystats.tex}

    Eight percent of hospitals in the sample have overlapping board members with an otherwise unaffiliated hospital. In Figure \ref{fig:connected_percent}, I show that this percentage remains relatively constant over time, with a slight decline from 129 affiliated hospitals in 2017 to 114 affiliated hospitals in 2021. I include 2022 values, but there are less total firms in the sample due to lag in tax forms being made available publicly.
    
    \begin{figure}[ht!]
        \centering
        \vspace{9mm}
        \caption{Percent of hospitals that have a board affiliation}
        \includegraphics[width=.8\textwidth]{Objects/connected_percent.pdf}
        \label{fig:connected_percent}
    \end{figure}
    
    As mentioned previously, common board members should be more likely to affect behavior within a hospital market, where hospitals compete for the same patient population. In Figure \ref{fig:connected_HRR_percent}, I show the prevalence of having shared board members within a hospital market. Of the 306 HRRs in the US, approximately 17\% have at least one pair of board affiliated hospitals within the market. This proportion decreases slightly over time, with 15\% of HRRs having connected hospitals in 2021, a nominal decrease of 8 markets losing hospitals with affiliation. This is likely due to increased system affiliations, which mechanically decreases board overlap since I do not include hospitals in the same system as eligible for having board overlap. 


    \begin{figure}[ht!]
        \centering
        \caption{Percent of HRRs containing board affiliated hospitals}
        \includegraphics[width=.8\textwidth]{Objects/connected_HRR_percent.pdf}
        \label{fig:connected_HRR_percent}
    \end{figure}

    Next, I present the geographic distribution of these pairs in Figure \ref{fig:connected_maps}. The blue lines represent hospitals that share a common board member, and the red dots are all other hospitals. Connected hospitals are distributed fairly evenly across the US, apart from relatively few connections on the west coast relative to the population. Over time, the connections remain fairly consistent geographically. Taken together, these show that board overlap is a prevalent occurrence among nonprofit hospitals in the US from 2017 onward. 

    \begin{figure}[ht!]
        \centering
        \caption{Geographic distribution of board affiliated hospitals over time}
        \includegraphics[width=\textwidth]{Objects/connected_maps.pdf}
        \label{fig:connected_maps}
    \end{figure}



    I now shift to examining characteristics of hospitals that share board members. I consider pairs of connected hospitals and characteristics of the pair in Table \ref{tab:hospital_pair_types}. Each hospital in a pair is general or specialty (I only consider adult hospitals for these categories), children or adult, and independent or belonging to a system, all characteristics in the AHA data. Additionally, the median bed size of all hospitals in the sample is 103, and I define hospitals as being either small or large depending on whether they fall below or above this median. 

    \import{Objects}{hospital_pair_types.tex}

    Two-thirds of board affiliated pairs are between two adult general hospitals, and 10\% of pairs are between a specialty and a general hospital. Thus, for the majority of hospitals that have board interlock, the hospital they share a board member with is a direct competitor. Additionally, there is significant variation in the ownership type of hospitals within affiliated pairs. Forty-five percent of pairs are between two non-system affiliated hospitals, and 36\% of pairs are between an independent hospital affiliated with a hospital belonging to a system. The lowest proportion of shared board member pairs is made up of two hospitals belonging to different systems. 
    
    There is also variation in whether the hospitals are the same size or not. Half of the hospital pairs consist of one small and one large hospital, and the other half are split evenly between two small or two large hospitals. However, the average difference in the number of beds of the two hospitals in a pair is very large: an average difference of 254 beds. This suggests that for hospitals of opposite size, the difference is drastic. The average distance between hospitals that share common board members is 35 miles, or 57 kilometers. In total, there are 217 hospitals that share a board member with at least one other hospital at some point in the sample. 

    \subsection{Correlations with Hospital Behaviors}

    I now merge the hospital board data to other publicly available data sets to examine descriptive differences in the hospital behaviors discussed in Section \ref{sec:hospbehaviors} for hospitals with and without board overlap. The broad categories of hospital behaviors that might be affected by affiliation through shared board members are price/quality, investment decisions, and differentiation, which are discussed in Section \ref{sec:hospbehaviors}. While I do not capture every aspect of hospital behavior, I select a number of variables that fall into these categories to examine. 

    First, from the Center for Medicare and Medicaid Services (CMS) Hospital Compare data, I merge information on each hospital's overall star rating. This rating combines measures of quality into one index that ranges from 1 to 5: readmission rates, mortality rates, safety, timeliness of care, effectiveness of care, and patient experience. This is a very aggregated measure, but may capture big-picture differences in quality.

    Next, I include data on different types of investments made into the hospital from the Healthcare Cost Reporting Information System (HCRIS) data. These variables include the dollar amounts spent on health information technology purchases, fixed equipment, and movable equipment. 

    There are a plethora of ways one could measure differentiation in hospitals. I focus on two categories of differentiation: service offerings and patient population. To measure different service offerings, I first construct a measure of concentration of beds devoted to categories of services in the AHA survey. For services $j$ in the list of all possible services\footnote{General medical and surgical (adult), General medical and surgical (pediatric), Obstetric care, Medical/surgical intensive care, Cardiac intensive care, Neonatal intensive care, Neonatal intermediate care, Pediatric intensive care beds, Burn care, Other Special Care, Other intensive care, Physical Rehabilitation care, Alcohol/drug abuse or dependency inpatient care, Psychiatric care, Skilled nursing care, Intermediate nursing care, Acute long term care, Other long-term care, Other care}, the concentration of bed allocation is defined as 

    $$concentration_{bed} = \sum_{j}\left(\frac{beds_j}{total beds}\right)^2.$$

    \noindent That is, a higher bed concentration indicates more specialization in bed allocation, where a lower concentration indicates more services spread out in bed allocations. Additionally, I include a second concentration measure that relies less on the facility and more on the conditions patients are treated with. The CMS Provider Utilization files contain measures of the percent of patients seen for different conditions. Therefore, for services $k$ in the list of conditions in the CMS data\footnote{Percent Of Beneficiaries Identified With Combined Cancer Flag For 6 Cancer Indicators, Percent Of Beneficiaries Identified With Chronic Obstructive Pulmonary Disease, Percent Of Beneficiaries Identified With Chronic Kidney Disease, Percent Of Beneficiaries Identified With Heart Failure And Non-Ischemic Heart Disease}, I define the concentration of patient conditions in each hospital as 

    $$HHI_{patient} = \sum_{k}\left(\frac{patients_k}{total patients}\right)^2.$$
    
    Finally, I include indicators for whether the hospital has a NICU or Cath Lab, both profitable services for hospitals to offer. I also consider differentiation by the patient population admitted to the hospital. There is information in the HCRIS data about how many discharges are for patients on Medicaid or Medicare. There is also information in the CMS Provider Utilization Files about the average risk score of Medicare patients seen in the hospital. I include all of these variables to capture something about the aggregate patient population being treated in different hospitals.

    In Table \ref{size_characteristics}, I present averages of all of these variables for four categories of hospitals: whether they are small (less than 103 beds) or large (greater than 103 beds), and whether they share board members with another competing hospital. I limit this sample to only general hospitals for relevant comparison. 

   \import{Objects}{size_characteristics}

   Small hospitals with and without shared board members have a very similar average star rating of 3.4, while large hospitals with shared board members have a slightly smaller average star rating than large hospitals without board overlap. In all categories of investment, small hospitals are relatively similar, with slightly more investment in health IT and fixed equipment for board affiliated small hospitals. Large hospitals are also generally similar in how they invest in health IT and fixed equipment relative to how much they spend on average, but hospitals with shared board members spend significantly more on movable equipment than hospitals without board overlap. 

   There do not seem to be meaningful differences in the concentration of services offered or conditions treated among hospitals with and without board overlap, whether they are small or large. Similarly, there are no clear differences in the patient composition of hospitals with and without board overlap. This is consistent even among the facilities of these hospitals, with no distinct differences in whether hospitals have a NICU or Cath lab based on board overlap. 

   Another important level of variation in this sample of hospitals could be whether they belong to a system or not, as this often dictates a number of the behaviors discussed. Therefore, I present a similar table of means for hospitals with and without board overlap that do or do not have a system affiliation in Table \ref{sys_characteristics}. 

   \import{Objects}{system_characteristics}

    Among hospitals that do not belong to a system, those without board overlap have a slightly higher quality rating than those without board overlap. However, the opposite is true among hospitals that do belong to a system, where those with board overlap have a higher quality rating. The most significant difference in investment is among hospitals that belong to a system for investment in movable equipment. Those with board overlap invest over 20 million in movable equipment, while those without board overlap only invest 8.3 million. The other comparisons of investment show no drastic differences. In-system and out-of-system hospitals both show higher levels of concentration of services when they do not have board overlap. However, the patient composition is similar within the categories of system affiliation. Independent hospitals that do not have board overlap are more likely to have a Cath lab, but otherwise hospitals are similar in in service offerings. 

    While these statistics do not establish a causal link between board overlap and hospital behavior, they highlight key differences across hospital categories. Notably, hospitals with overlapping board members exhibit distinct patterns in quality and investment decisions. These differences underscore the need for further research to explore board overlap as a potential mechanism for affiliation and behavioral changes in health care.
    
    \section{Conclusion}


    This study contributes to the growing literature on hospital governance and market structure by examining the extent and implications of board member overlap across hospitals. By identifying the prevalence of shared board membership, I provide new insights into how governance ties may influence competition, decision-making, and resource allocation in the hospital sector. This paper sheds light on the role of board interlocks in potentially shaping hospital behavior, offering an important perspective on the non-ownership mechanisms that can link hospitals within the same market.

    Using a comprehensive dataset of board members in nonprofit hospitals from 2017-2022, I document the characteristics of hospitals with shared board members and analyze how these overlaps have evolved over time. I identify connections between hospitals and examine the patterns of overlap by hospital type, market, and geographic region. The results indicate that a significant share of hospitals are connected through board interlocks, and a significant portion of hospital markets contain board overlap within the market. Further, I show geographic distributions of overlap and characterize connected hospital pairs by important characteristics such as ownership, size, and specialty. 

    These findings hold important implications for policymakers and future research. Regulators and antitrust authorities should consider the potential competitive effects of board member overlap when assessing hospital market dynamics, particularly in regions with high consolidation or limited provider choice. Future research could further investigate how these governance connections influence hospital pricing, quality of care, and patient outcomes. Expanding this work to explore causal relationships and the broader economic consequences of board interlocks would provide valuable insights for competition in health care.
  

    \newpage


    \printbibliography


    

    

    

    

    

	
	
	


\end{document}